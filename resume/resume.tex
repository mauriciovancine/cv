\documentclass{resume}

\usepackage[left=0.75in,top=0.6in,right=0.75in,bottom=0.6in]{geometry} % Document margins
\newcommand{\tab}[1]{\hspace{.2667\textwidth}\rlap{#1}}
\newcommand{\itab}[1]{\hspace{0em}\rlap{#1}}
\name{Maurício H. Vancine} % Your name
\address{São Paulo State University (UNESP), Rio Claro, SP, Brazil}
\address{mauricio.vancine@gmail.com, +55-19-993-340-549, \href{https://mauriciovancine.github.io/}{mauriciovancine.github.io}}
\address{Ecologist and Doctoral Student in Ecology, Evolution and Biodiversity}

\begin{document}

%----------------------------------------------------------------------------------------
%	Research Interests
%----------------------------------------------------------------------------------------

\begin{rSection}{Research Interests}

Habitat loss and fragmentation, landscape metrics, species distribution modeling (SDM), community ecology, amphibian ecology, conservation, teaching statistics and R language

\end{rSection}

%----------------------------------------------------------------------------------------
%	Education
%----------------------------------------------------------------------------------------

\begin{rSection}{Education}

{\bf São Paulo State University (UNESP)} \hfill {\em Mar 2020 - Mar 2024 (expected)} \\ 
Ph.D., Graduate Program in Ecology, Evolution and Biodiversity \hfill { GPA: 4.00 } \\ 
Thesis: Landscape structure as a predictor of taxonomic and functional diversity of amphibians in the Atlantic Forest \\ 
Advisor: Prof. Dr. Milton Cezar Ribeiro

{\bf São Paulo State University (UNESP)} \hfill {\em Mar 2016 - Jul 2018} \\ 
M.S., Graduate Program in Biological Science (Zoology)\hfill { GPA: 4.00 } \\ 
Thesis: Diversity, distribution and effect of climate change on Atlantic Forest amphibian communities \href{https://repositorio.unesp.br/handle/11449/154993}{(UNESP Libraries)} \\ 
Advisor: Prof. Dr. Célio Fernando Baptista Haddad

{\bf São Paulo State University (UNESP)} \hfill {\em Mar 2011 - Mar 2015} \\ 
B.S. in Ecology\hfill { GPA: 3.43 } \\ 
Thesis: Effect of fragmentation on the persistence of anuran amphibians (Amphibia: Anura) within the Atlantic Forest \href{https://repositorio.unesp.br/handle/11449/138991}{(UNESP Libraries)} \\ 
Advisor: Prof. Dr. Milton Cezar Ribeiro

\end{rSection}

%--------------------------------------------------------------------------------
%	Publications 
%--------------------------------------------------------------------------------

\begin{rSection}{Publications (selected)}

{\bf Peer reviewed}

1. Beca G, \underline{\bf Vancine MH}, Carvalho CS, Pedrosa F, Alves RSC, Buscariol D, Peres CA, Ribeiro MC, Galetti M. 2017. {\bf High mammal species turnover in forest patches immersed in biofuel plantations}. {\it Biological Conservation} 210:352–359. \href{https://doi.org/10.1016/j.biocon.2017.02.033}{10.1016/j.biocon.2017.02.033}

2. \underline{\bf Vancine MH}, Duarte KS, Souza YS, Giovanelli JGR, Sobrinho PMM, López A, Bovo RP, Maffei F, Lion MB, Ribeiro-Júnior JW, Brassaloti R, Ortiz C, Sawakuchi HO, Forti LR, Cacciali P, Bertoluci J, Haddad CFB, Ribeiro MC. 2018. {\bf ATLANTIC AMPHIBIANS: a data set of amphibian communities from the Atlantic Forests of South America}. {\it Ecology} 99(7):1692–1692. \href{https://doi.org/10.1002/ecy.2392}{10.1002/ecy.2392}

3. Marjakangas E, Abrego N, Grøtan V, Lima RAF, Bello C, Bovendorp RS, Culot L, Hasui E, Muylaert RL, Lima F, Niebuhr B, Oliveira AA, Pereira L, Prado I, Stevens RD, \underline{\bf Vancine MH}, Ribeiro MC, Galetti M, Ovaskainen O. 2020. {\bf Fragmented tropical forests lose mutualistic plant-animal interactions}. {\it Diversity and Distributions} 26(2):154-168. \href{https://doi.org/10.1111/ddi.13010}{10.1111/ddi.13010}

4. Muylaert RL, Kingston T, Luo J, \underline{\bf Vancine MH}, Galli N, Carlson CJ, John RS, Rulli MC, Hayman DT. 2022. {\bf Present and future distribution of bat hosts of sarbecoviruses: implications for conservation and public health}. {\it Proceedings of the Royal Society B} 289(1975):20220397. \href{https://doi.org/10.1098/rspb.2022.0397}{10.1098/rspb.2022.0397}

5. Monteiro ECS, Pizo MA, \underline{\bf Vancine MH}, Ribeiro MC. 2022. {\bf Forest cover and connectivity have pervasive effects on the maintenance of evolutionary distinct interactions in seed dispersal networks}. {\it Oikos} 2022(2):oik.08240. \href{https://doi.org/10.1111/oik.08240}{10.1111/oik.08240}

{\bf Book}

1. Da Silva FR, Gonçalves-Souza T, Paterno GB, Provete DB, \underline{\bf Vancine MH}. 2022. {\bf Análises Ecológicas no R}. Nupeea: Recife, PE, Canal 6: São Paulo. 640 p. ISBN 978-85-7917-564-0. \href{https://analises-ecologicas.com/}{link}

{\bf For all publications, citations, and reviews}: \href{https://orcid.org/0000-0001-9650-7575}{ORCID}, \href{https://www.webofscience.com/wos/author/record/837504}{Web of Science}, \href{https://www.scopus.com/authid/detail.uri?authorId=57193451888}{Scopus}, \href{https://scholar.google.com/citations?user=i-2xZBQAAAAJ}{Google Scholar}

\end{rSection}

%--------------------------------------------------------------------------------
%	Manuscripts in review
%--------------------------------------------------------------------------------

\begin{rSection}{Manuscripts in review }

1. Teixeira JVS, Bonfim FCG, \underline{\bf Vancine MH}, Ribeiro MC, Oliveira LC. Effect
of landscape attributes at multiple scales on the occurrence of the threatened golden-
headed lion tamarin, Leontopithecus chrysomelas Kuhl, 1820 (Primates, Callitrichidae).
{\it American Journal of Primatology}.

2. Tonetti V, Bocalini F, Schunck F, \underline{\bf Vancine MH}, Butti M, Ribeiro MC, Pizo M,
Balmford A. The Protected Areas network may be inefficient to cover biodiversity in a
fragmented tropical hotspot under different climate scenarios. {\it Perspective in Ecology and
Conservation}.

3. Anunciação PA, Ernst R, Martello F, \underline{\bf Vancine MH}, Ribeiro MC, Carvalho LMT. Climate-driven loss of taxonomic and functional richness in Atlantic Forest anurans. {\it Perspective in Ecology and Conservation}.

\end{rSection}

%--------------------------------------------------------------------------------
%	Grants and Fellowships
%--------------------------------------------------------------------------------

\begin{rSection}{Grants and Fellowships} 

{\bf Doctoral Graduate Scholarship}{, São Paulo Research Foundation (FAPESP)
} \hfill{\em 2022-2024} \\ 
Grant: \href{https://bv.fapesp.br/en/bolsas/203713/landscape-structure-as-a-predictor-of-taxonomic-and-functional-diversity-of-amphibians-in-the-atlant/}{2022/01899-6} \\Value: BRL 91,002.24

{\bf Master Graduate Scholarship}{, São Paulo Research Foundation (FAPESP)
} \hfill{\em 2017-2018} \\ 
Grant: \href{https://bv.fapesp.br/en/bolsas/172826/effect-of-landscape-modifications-and-climate-changes-on-the-persistence-of-amphibians-in-the-atlant/}{2017/09676-8} \\ Value: BRL 16,248.54

{\bf Undergraduate Research Scholarship}{, São Paulo Research Foundation (FAPESP)
} \hfill{\em 2013-2015} \\ 
Grant: \href{https://bv.fapesp.br/en/bolsas/142421/effect-of-fragmentation-on-the-persistence-of-anuran-amphibians-amphibia-anura-within-the-atlanti/}{2013/02883-7} \\ Value: BRL 10,539.87

\end{rSection}

%--------------------------------------------------------------------------------
%	Awards and Honors
%--------------------------------------------------------------------------------

\begin{rSection}{Awards and Honors} 

{High Academic Performance Award, São Paulo State University (UNESP)} \hfill{\em 2015}

\end{rSection}

%--------------------------------------------------------------------------------
%	Research Experience
%--------------------------------------------------------------------------------

\begin{rSection}{Research Experience}

{\bf Doctoral research} \hfill{\em 2020-currently} \\ 
Title: Landscape structure as a predictor of taxonomic and functional diversity of amphibians in the Atlantic Forest \\ 
Aim: analyzing the structure of landscapes throughout the Atlantic Forest; analyze how landscape structure affects the taxonomic and functional diversity of amphibian communities; and analyze this same question of species-specific mode, in addition to analyzing co-occurrences using JSDMs.

{\bf Master research} \hfill{\em 2016-2018} \\ 
Title: Effect of Landscape Modifications and Climate Changes on the Persistence of Amphibians in the Atlantic Forest \\ 
Aim: to present an assessment of the surveys of the amphibian communities for the Atlantic Forest; to investigate how habitat loss and fragmentation affect the persistence of amphibians; and to investigate how climate change affect the future persistence of the genus {\it Brachycephalus}.

{\bf Undergraduate research} \hfill{\em 2013-2015} \\ 
Title: Effect of fragmentation on the persistence of anuran amphibians (Amphibia: Anura) within the Atlantic Forest \\ 
Aim: evaluate the relative contribution of landscape indices (percentage of forest cover, connectivity, relief and urban proximity) to the persistence of species using species distribution modeling.

\end{rSection}

%--------------------------------------------------------------------------------
%	Teaching Experience
%--------------------------------------------------------------------------------

\begin{rSection}{Teaching Experience}

{\bf Graduation (300 h)}

{\bf Introduction to the use of geospatial data in R (60 h)} \hfill{\em 2021} \\ 
Invited teacher, Ecology, Biodiversity and Evolution Graduate Program, São Paulo State University (UNESP)

{\bf Introduction to the use of geospatial data in R (60 h)} \hfill{\em 2020} \\ 
Invited teacher, Ecology, Biodiversity and Evolution Graduate Program, São Paulo State University (UNESP)

{\bf Introduction to Geoprocessing for Ethnobiology and Biodiversity Conservation (45 h)} \hfill{\em 2019} \\ 
Invited teacher, Ethnobiology and Nature Conservation Graduate Program, Rural Federal University of Pernambuco (UFRPE)

{\bf Ecological Niche Modeling: theory and practice (60 h)} \hfill{\em 2017} \\ 
Teaching assistant, Ecology and Biodiversity Graduate Program, São Paulo State University (UNESP)

{\bf Ecological Niche Modeling: theory and practice (45h)} \hfill{\em 2016} \\ 
Teaching assistant, Ecology Graduate Program, The University of Campinas (UNICAMP)

{\bf Ecological Niche Modeling in R (30 h)} \hfill{\em 2016} \\ 
Teaching assistant, Ecology and Biodiversity Graduate Program, Graduate Program, São Paulo State University (UNESP)

{\bf Undergraduate (102 h)}

{\bf Statistical Models in Ecology (30 h)} \hfill{\em 2020} \\ 
Teacher, Ecology Undergraduate, São Paulo State University (UNESP)

{\bf Statistical Models in Ecology (12 h)} \hfill{\em 2018} \\ 
Teaching assistant, Ecology Undergraduate, São Paulo State University (UNESP)

{\bf Quantitative Ecology (60 h)} \hfill{\em 2015} \\ 
Teaching assistant, Ecology Undergraduate, São Paulo State University (UNESP)

\end{rSection}

%--------------------------------------------------------------------------------
%   Professional Experience
%--------------------------------------------------------------------------------

\begin{rSection}{Professional Experience}

{\bf Environmental consultant} \hfill{\em 2023} \\ 
Project: Impactos das metas de recuperação de pastagens degradadas do Plano ABC+ no Brasil \\ 
Role: Landscape Ecology and Ecological Data Analyst \\ 
Company: Seleção Natural, Piracicaba, SP, Brazil

{\bf Environmental consultant} \hfill{\em 2021-2023} \\ 
Project: Plano de ação da Suzano para conservação de primatas e palmeiras \\ 
Role: Ecological Modeling, Ecological Data Analyst, Species Distribution Modeling, Landscape Ecology \\ 
Company: Seleção Natural, Piracicaba, SP, Brazil

{\bf Environmental consultant} \hfill{\em 2021} \\ 
Project: Plano de Manejo do Parque Estadual Marinho do Parcel de Manuel Luís – MA \\ 
Role: Ecological Modeling and Ecological Data Analyst\\ 
Company: Seleção Natural, Piracicaba, SP, Brazil

{\bf Environmental consultant} \hfill{\em 2021} \\ 
Project: Avaliação de oferta de serviços ambientais relacionados à Biodiversidade (modelagem de biodiversidade para a plataforma PSA TNC) \\ 
Role: Landscape Ecology and Ecological Data Analyst \\ 
Company: Seleção Natural, Piracicaba, SP, Brazil

{\bf Environmental consultant} \hfill{\em 2020-2021} \\ 
Project: Delineamento espacial de corredores ecológicos para subsidiar o desenvolvimento da Meta de Longo Prazo - MLP de Biodiversidade da Suzano S.A. \\ 
Role: Landscape Ecology and Ecological Data Analyst \\ 
Company: Seleção Natural, Piracicaba, SP, Brazil

{\bf Environmental consultant} \hfill{\em 2019-2021} \\ 
Project: Monitoramento regional da fauna de vertebrados terrestres (avifauna, herpetofauna e mastofauna) nas áreas de influência da Companhia Brasileira de Alumínio - Zona da Mata, Minas Gerais \\ 
Role: Species Distribution Modeling and Ecological Data Analyst \\ 
Company: PROECO Ambiental Poços de Caldas, MG, Brazil

{\bf Environmental consultant} \hfill{\em 2018-2019} \\ 
Project: Onde estão e quantos são os gatos, mutuns-de-bico-vermelho e os macacos ameaçados da região da Barra do Tijuípe \\ 
Role: Ecological Data Analyst \\ 
Company: Seleção Natural, Piracicaba, SP, Brazil

{\bf Environmental consultant} \hfill{\em 2018-2019} \\ 
Project: Modelagem da biodiversidade, conectividade e regeneração florestal da Bacia do Rio Doce \\ 
Role: Species Distribution Modeling, Ecological Data Analyst and Landscape Ecology\\ 
Company: Aquaflora and Lactec, Curitiba, PR, Brazil

{\bf Environmental consultant} \hfill{\em 2018-2019} \\ 
Project: Programa de monitoramento da biodiversidade da Estanho Rondônia S/A para atividade de mineração frentes de lavra Taboquinha, Village e Serra da Onça C \\ 
Role: Ecological Data Analyst \\ 
Company: Seleção Natural, Piracicaba, SP, Brazil

{\bf Environmental consultant} \hfill{\em 2018} \\ 
Project: Aquisição, gerenciamento e processamento de imagens de satélite de vários sensores para o projeto USAID PEER \\ 
Role: Shiny App \\ 
Company: Associação de Levantamento Florestal do Amazonas Manaus, AM, Brazil

{\bf Environmental consultant} \hfill{\em 2016-2017} \\ 
Project: Modelagem de regeneração natural potencial da Mata Atlântica com base na estrutura da paisagem e na movimentação da fauna frugívora \\ 
Role: Landscape Ecology and SIG specialist \\ 
Company: ECOnecta Rio Claro, SP, Brazil

{\bf Environmental consultant} \hfill{\em 2015-2016} \\
Project: Inventário da Herpetofauna, referente ao Relatório Ambiental Preliminar (RAP) da Mineração Darcy, São Simão - SP \\ 
Role: Herpetology field assistant \\ 
Company: Geoinform Rio Claro, SP, Brazil

\end{rSection}

%--------------------------------------------------------------------------------
%   Professional Societies 
%--------------------------------------------------------------------------------

\begin{rSection}{Professional Societies}

Associação Brasileira de Ecólogos (ABE)
\\ Associação Brasileira de Ciência Ecológica e Conservação (ABECO)

\end{rSection}

%--------------------------------------------------------------------------------
%   Academic Advisories
%--------------------------------------------------------------------------------

\begin{rSection}{Academic Advisories}

{\bf Lucas de Souza Almeida} \hfill{\em 2022} \\ 
{\it Conhecimento atual, distribuição potencial e conservação de Cecílias (Amphibia: Gymnophiona)} \\
Coadvidor, Master thesis, Ecology, Biodiversity and Evolution Graduate Program, São Paulo State University (UNESP)

{\bf Bruno Eduardo Ribeiro Silva} \hfill{\em 2022} \\ 
{\it Comparação da distribuição de espécies inferidos por Modelos de Nicho Ecológicos e mapa de especialistas da IUCN para anfíbios anuros da América do Sul} \\
Undergraduate thesis, Ecology, São Paulo State University (UNESP)

{\bf Helena Thereza Carvalho de Oliveira} \hfill{\em 2021} \\ 
{\it Distribuição do padrão reprodutivo em comunidades de anuros na Mata Atlântica} \\
Undergraduate thesis, Biological Sscience, São Paulo State University (UNESP)

\end{rSection}

%--------------------------------------------------------------------------------
%   Services
%--------------------------------------------------------------------------------

\begin{rSection}{Services}

Reviewer for: {\it Zoologia}, {\it PLOS One}, {\it Biological Invasions}, {\it Hydrobiologia}, {\it Scientific Data}, {\it Diversity and Distributions}, {\it Papéis Avulsos de Zoologia}

\end{rSection}

%--------------------------------------------------------------------------------
%	skills
%--------------------------------------------------------------------------------

\begin{rSection}{Skills}

{\bf Programming languages}
\\ R (advanced), tidyverse (advanced), markdown (advanced), shiny (basic), LaTeX (basic), git (basic), Python (basic), JavaScript (basic), shell/bash (basic)

{\bf R packages}
\\ {\it amphiBR}: dataset from the official publication of the List of Amphibians in Brazil published by the Brazilian Society of Herpetology
\\ Authors: Paulo Barros de Abreu Junior, \underline{\bf Maurício Humberto Vancine} e Diogo B. Provete
\\ Site: \href{https://paulobarros.github.io/amphiBR}{paulobarros.github.io/amphiBR}
\\ {\it LSmetrics}: multiscale calculation and spatialization of landscape metrics using GRASS GIS  
\\ Authors: Bernardo Niebuhr, \underline{\bf Maurício Humberto Vancine}, Renata de Lara Muylaert, John Wesley Ribeiro e Milton Cezar Ribeiro
\\ Site: \href{https://mauriciovancine.github.io/lsmetrics}{mauriciovancine.github.io/lsmetrics}

{\bf Statistical knowledge}
\\ {\it Descriptive statistics}: tabular data manipulation (advanced), spatial data manipulation (advanced), tabular data visualization (advanced), spatial data visualization (advanced)
\\ {\it Inferential statistics}: frequentist (advanced), likelihood (intermediate) and bayesian (basic)

{\bf Softwares}
\\ QGIS (advanced), GRASS GIS (advanced), GNU/Linux (intermediate)

{\bf Languages}
\\ English (intermediate)
\\ Spanish (basic)
\\ Portuguese (native speaker)

\end{rSection}

%--------------------------------------------------------------------------------
%   References
%--------------------------------------------------------------------------------

\begin{rSection}{References}
{\bf Prof. Dr. Milton Cezar Ribeiro from São Paulo State University (UNESP), Brazil} 
\\ milton.c.ribeiro@unesp.br +55-19-3526-9680

{\bf Prof. Dr. Célio F. B. Haddad from São Paulo State University (UNESP), Brazil} 
\\ haddad1000@gmail.com +55-19-3526-4302

{\bf Prof. Dr. Carlos Henrique Grohmann from University of São Paulo (USP), Brazil} 
\\ guano@usp.br +55-11-3091-2627

{\bf Prof. Dr. Thiago Sanna Freire Silva from University of Stirling, UK} 
\\ thiago.sf.silva@stir.ac.uk +44-798-8888-6891

{\bf Dr. Thiago Gonçalves-Souza from University of Michigan, USA} 
\\ tgoncalv@umich.edu +1-734-596-0508

{\bf Dr. Renata Lara Muylaert from Massey University, New Zealand} 
\\ R.deLaraMuylaert@massey.ac.nz +64-06-356-9099 ext. 85217

{\bf Dr. João Gabriel Ribeiro Giovanelli from Seleção Natural, Brazil} 
\\ joao@selecaonatural.net +55-19-99103-7348

\end{rSection}

\end{document}----------------------------