\documentclass{resume}

\usepackage[left=0.75in,top=0.6in,right=0.75in,bottom=0.75in]{geometry}
\usepackage[utf8]{inputenc}
\usepackage[T1]{fontenc}

\name{Maurício H. Vancine}
\address{State University of Campinas (Unicamp), Campinas, SP, Brazil}
\address{
\href{mailto:}{mauricio.vancine@gmail.com} 
$\ \bullet\ $ 
\href{https://github.com/mauriciovancine}{GitHub} 
$\ \bullet\ $ 
\href{https://www.linkedin.com/in/mauricio-vancine/}{LinkedIn} 
$\ \bullet\ $ 
\href{https://mauriciovancine.github.io}{Website}}
\address{Ecologist, Data Scientist, and PhD in Ecology}



% cor dos links
\usepackage{xcolor} % Pacote para manipulação de cores
\usepackage{hyperref} % Pacote para links

\hypersetup{
    colorlinks=true, % Ativa a coloração dos links
    linkcolor=blue, % Cor dos links internos
    urlcolor=blue,  % Cor dos links externos (URLs)
    citecolor=blue  % Cor das citações
}

\begin{document}

%------------------------------------------------------------------------
%	Career summary
%------------------------------------------------------------------------

\begin{rSection}{Career summary}
I am an Ecologist (2014), Master's in Zoology (2018), and PhD in Ecology (2024) all from Universidade Estadual Paulista, Rio Claro/SP, Brazil. I performed over 700 hours of additional education at Brazilian and international institutions, focusing on statistical and ecological modeling. 

My expertise encompasses spatial ecology, landscape ecology, ecological modeling, species distribution modeling (SDM), ecological and spatial data analysis, impacts of habitat loss and fragmentation on biodiversity, amphibian ecology, teaching the R programming language. 

Since 2017, I have authored over 30 articles published in prominent Ecology journals. These publications primarily showcase the outcomes of my research on the effects of landscape structure and climate change on biodiversity in South America, with a focus on the Atlantic Forest hotspot.

Currently, I am a Postdoctoral researcher at \href{https://www.mathiasmpires.net.br/index.html}{\underline{lab.exe}} at \href{https://unicamp.br/}{\underline{Unicamp}} and my research involves integrating landscape metrics and species distribution modeling (SDM) to assess the impacts of edge effects, fires, and climate change on the distribution and interactions of animals and plants across multiple spatial scales in the Amazon.

I am actively engaged in the development of tools to calculate landscape metrics for large spatial extents using the R language and GRASS GIS. In addition, I have dedicated over 500 hours to teaching and promoting the use of the R language through workshops, undergraduate and graduate courses, and the publication of the book 'Análises Ecológicas no R'.
\end{rSection}

%------------------------------------------------------------------------
%	Academic expertise
%------------------------------------------------------------------------

\begin{rSection}{Academic expertise}

Spatial ecology, landscape ecology, ecological modeling, species distribution modeling (SDM), geoprocessing, amphibians, conservation, teaching, R language

\end{rSection}

%------------------------------------------------------------------------
%	Research interests
%------------------------------------------------------------------------

\begin{rSection}{Research interests}

Habitat loss and fragmentation, landscape metrics, species distribution modeling (SDM), community ecology, amphibian ecology, conservation biology, teaching R language and statistics

\end{rSection}

%------------------------------------------------------------------------
%	Education
%------------------------------------------------------------------------

\begin{rSection}{Education}

{\bf São Paulo State University (UNESP)} \hfill {\em Mar 2011 - Mar 2015}\\
B.S. in Ecology\hfill { GPA: 3.43 }\\
Thesis: Effect of fragmentation on the persistence of anuran amphibians (Amphibia: Anura) within the Atlantic Forest\\
Library: \href{http://hdl.handle.net/11449/138991}{\underline{link}}\\
Advisor: Prof. Dr. Milton Cezar Ribeiro

{\bf São Paulo State University (UNESP)} \hfill {\em Mar 2016 - Jul 2018}\\
M.S., Graduate Program in Biological Science (Zoology)\hfill { GPA: 4.00 }\\
Thesis: Diversity, distribution and effect of climate change on Atlantic Forest amphibian communities\\
Library: \href{http://hdl.handle.net/11449/154993}{\underline{link}}\\
Advisor: Prof. Dr. Célio Fernando Baptista Haddad

{\bf São Paulo State University (UNESP)} \hfill {\em Mar 2020 - Jul 2024}\\
Ph.D., Graduate Program in Ecology, Evolution and Biodiversity \hfill { GPA: 4.00 }\\Dissertation: Landscape structure as a predictor of taxonomic and functional diversity of amphibians in the Atlantic Forest\\
Library: \href{https://hdl.handle.net/11449/256726}{\underline{link}}\\
Advisor: Prof. Dr. Milton Cezar Ribeiro

\end{rSection}

%------------------------------------------------------------------------
%	Complementary education
%------------------------------------------------------------------------

\begin{rSection}{Complementary education}

\begin{itemize} 

\item {\bf Ecological data analysis with R (40 h)} \hfill {\em 2011}\\
São Paulo State University (UNESP), Rio Claro, SP, Brazil

\item {\bf Biology and conservation of amphibians and reptiles (44 h)} \hfill {\em 2015}\\
Instituto Boitatá, IBEC, Alfenas, MG, Brazil, \href{https://www.institutoboitata.org/}{\underline{link}}

\item {\bf V Southern-Summer School on Mathematical Biology (53 h)} \hfill {\em 2016}\\
São Paulo State University (UNESP), São Paulo, SP, Brazil, \href{https://www.ictp-saifr.org/v-southern-summer-school-on-mathematical-biology}{\underline{link}} 

\item {\bf Geoprocessing with GRASS GIS (24 h)} \hfill {\em 2016}\\
São Paulo State University (UNESP), Rio Claro, SP, Brazil

\item {\bf Introduction to Hierarchical Modeling (45 h)} \hfill {\em 2019}\\
Federal University of Rio Grande do Sul (UFRGS), Porto Alegre, RS, Brazil, \href{http://ferrazlab.org/graduate}{\underline{link}}

\item {\bf School on Community Ecology: from patterns to principles (60 h)} \hfill {\em 2020}\\
São Paulo State University (UNESP), São Paulo, SP, Brazil, \href{https://www.ictp-saifr.org/community-ecology-from-patterns-to-principles}{\underline{link}}

\item {\bf Hierarchical Modelling of Species Communities with the R-package Hmsc (25 h)} \hfill {\em 2020}\\
University of Helsinki (Online), Helsinki, Finland, \href{https://www.helsinki.fi/en/researchgroups/statistical-ecology/hmsc}{\underline{link}}

\item {\bf Joint Species Distribution Modelling with HMSC (45 h)} \hfill {\em 2022}\\
University of Jyväskylä (Online), Jyväskylä, Finlândia, \href{https://www.helsinki.fi/en/researchgroups/statistical-ecology/software/hmsc}{\underline{link}}

\item {\bf Data Science Journey (330 h)} \hfill {\em 2022}\\
Ômega Data Science (Online), Curitiba, Paraná, \href{https://escola.omegadatascience.com.br}{\underline{link}}
\end{itemize} 

\end{rSection}

%------------------------------------------------------------------------
%	Professional experience
%------------------------------------------------------------------------

\begin{rSection}{Professional experience}
\begin{itemize}
\item {\bf Intern (300 h)} \hfill{\em 2014}\\
Compulsory curricular internship focused in Species Distribution Modeling (SDM\\
Institution: Luiz de Queiroz College of Agriculture, University of São Paulo (USP), Ecology, Management and Conservation of Wild Fauna Lab (LEMaC)\\
Supervisor: Prof. Dr. Katia Maria Paschoaletto Micchi de Barros Ferraz

\item {\bf Research assistant (2000 h)} \hfill{\em 2015-2016}\\
Statistical and spatial analyses of mammal and ants biodiversity in the Atlantic Forest Biome\\
Institution: São Paulo State University (UNESP), Spatial Ecology and Conservation Lab (LEEC)\\
Supervisor: Prof. Dr. Milton Cezar Ribeiro

\item {\bf Postdoctoral researcher} \hfill{\em 2024-actual}\\
Impacts of edge effects, fire, and climate change on vegetation composition and functional diversity across spatial scales in the Amazon rainforest\\
Institution: State University of Campinas (Unicamp), Ecology Extintion Lab (lab.exe)\\
Supervisor: Prof. Dr. Mathias Pires Mistretta
\end{itemize}
\end{rSection}

%------------------------------------------------------------------------
%	Research experience
%------------------------------------------------------------------------

\begin{rSection}{Research experience}
\begin{itemize}
\item {\bf Undergraduate research} \hfill{\em 2013-2015}\\
Title: Effect of fragmentation on the persistence of anuran amphibians (Amphibia: Anura) within the Atlantic Forest\\
Aim: To evaluate the relative contribution of landscape indices (percentage of forest cover, connectivity, relief and urban proximity) to the persistence of species using species distribution modeling.

\item {\bf Master research} \hfill{\em 2016-2018}\\
Title: Effect of Landscape Modifications and Climate Changes on the Persistence of Amphibians in the Atlantic Forest\\
Aim: To present an assessment of the surveys of the amphibian communities for the Atlantic Forest; to investigate how habitat loss and fragmentation affect the persistence of amphibians; and to investigate how climate change affect the future persistence of the genus {\it Brachycephalus}.

\item {\bf Doctoral research} \hfill{\em 2020-2024}\\
Title: Landscape structure as a predictor of taxonomic and functional diversity of amphibians in the Atlantic Forest\\
Aim: To analyze the structure of landscapes throughout the Atlantic Forest; to analyze how landscape structure affects the taxonomic and functional diversity of amphibian communities; and to analyze this same question of species-specific mode, in addition to analyzing co-occurrences using JSDMs.

\item {\bf Postdoctoral research} \hfill{\em 2024-actual}\\
Title: Impacts of edge effects, fire, and climate change on vegetation composition and functional diversity across spatial scales in the Amazon rainforest\\
Aim: Understand how edge effects, fires and climate change acting at local scales can influence the distribution and composition patterns of vegetation at different spatial scales in the Amazon biome, also affecting the potential distribution of vertebrate fauna. In addition, we will investigate how plant functional attributes respond to these factors, with implications for the functional diversity of plant and animal communities.
\end{itemize}
\end{rSection}

%------------------------------------------------------------------------
%	Funding
%------------------------------------------------------------------------

\begin{rSection}{Funding} 
\begin{itemize}
\item {\bf Undergraduate Research Scholarship}{, São Paulo Research Foundation (FAPESP)
} \hfill{\em 2013-2015}\\
Grant: \href{https://bv.fapesp.br/en/bolsas/142421/effect-of-fragmentation-on-the-persistence-of-anuran-amphibians-amphibia-anura-within-the-atlanti/}{\underline{2013/02883-7}}\\
Value: BRL 10,539.87

\item {\bf Master Graduate Scholarship}{, São Paulo Research Foundation (FAPESP)
} \hfill{\em 2017-2018}\\
Grant: \href{https://bv.fapesp.br/en/bolsas/172826/effect-of-landscape-modifications-and-climate-changes-on-the-persistence-of-amphibians-in-the-atlant/}{\underline{2017/09676-8}}\\
Value: BRL 16,248.54

\item {\bf Doctoral Graduate Scholarship}{, São Paulo Research Foundation (FAPESP)
} \hfill{\em 2022-2024}\\
Grant: \href{https://bv.fapesp.br/en/bolsas/203713/landscape-structure-as-a-predictor-of-taxonomic-and-functional-diversity-of-amphibians-in-the-atlant/}{\underline{2022/01899-6}}\\
Value: BRL 91,002.24

\item {\bf Postdoctoral Scholarship}{, São Paulo Research Foundation (FAPESP)
} \hfill{\em 2024-actual}\\
Grant: \href{}{\underline{2024/19865-6}}\\
Value: BRL 316,800.00
\end{itemize}
\end{rSection}

%------------------------------------------------------------------------
%	Awards and Honors
%------------------------------------------------------------------------

\begin{rSection}{Awards and honors} 
\begin{itemize}
\item {High Academic Performance Award, São Paulo State University (UNESP)} \hfill{\em 2015}
\item {The Harry R. Painton Award, American Ornithological Society} \hfill{\em 2023}
\end{itemize}
\end{rSection}

%-------------------------------------------------------------------------
%	Publications 
%-------------------------------------------------------------------------

\begin{rSection}{Publications}

{\bf Peer-reviewed (most relevant)}

\begin{enumerate} 

\item {\bf Vancine MH}, Muylaert RL, Niebuhr BB, Oshima JEF, Tonetti V, Bernardo R, De Angelo C, Rosa MR, Grohmann CH, Ribeiro MC. {\bf The Atlantic Forest of South America: spatiotemporal dynamics of remaining vegetation and implications for conservation}. {\it Biological Conservation} 291:110499. \href{https://doi.org/10.1016/j.biocon.2024.110499}{\underline{10.1016/j.biocon.2024.110499}}

\item Beca G, {\bf Vancine MH}, Carvalho CS, Pedrosa F, Alves RSC, Buscariol D, Peres CA, Ribeiro MC, Galetti M. 2017. {\bf High mammal species turnover in forest patches immersed in biofuel plantations}. {\it Biological Conservation} 210:352-359. \href{https://doi.org/10.1016/j.biocon.2017.02.033}{\underline{10.1016/j.biocon.2017.02.033}}

\item Marjakangas E, Abrego N, Grøtan V, Lima RAF, Bello C, Bovendorp RS, Culot L, Hasui E, Muylaert RL, Lima F, Niebuhr B, Oliveira AA, Pereira L, Prado I, Stevens RD, {\bf Vancine MH}, Ribeiro MC, Galetti M, Ovaskainen O. 2020. {\bf Fragmented tropical forests lose mutualistic plant-animal interactions}. {\it Diversity and Distributions} 26(2):154-168. \href{https://doi.org/10.1111/ddi.13010}{\underline{10.1111/ddi.13010}}

\end{enumerate} 

{\bf Peer-reviewed (\today)}

\begin{enumerate} 
\item Beca G, {\bf Vancine MH}, Carvalho CS, Pedrosa F, Alves RSC, Buscariol D, Peres CA, Ribeiro MC, Galetti M. 2017. {\bf High mammal species turnover in forest patches immersed in biofuel plantations}. {\it Biological Conservation} 210:352-359. \href{https://doi.org/10.1016/j.biocon.2017.02.033}{\underline{10.1016/j.biocon.2017.02.033}}

\item de Castro Pena JC, Goulart F, Fernandes GW, Hoffmann D, Leite FS, dos Santos NB, Soares-Filho B, Sobral-Souza T, {\bf Vancine MH}, Rodrigues M. 2017. {\bf Impacts of mining activities on the potential geographic distribution of eastern Brazil mountaintop endemic species}. {\it Perspectives in Ecology and Conservation} 15(3):172-178. \href{https://doi.org/10.1016/j.pecon.2017.07.005}{\underline{10.1016/j.pecon.2017.07.005}}

\item Regolin AL, Cherem JJ, Graipel ME, Bogoni JA, Ribeiro JW, {\bf Vancine MH}, Castilho PVD. 2017. {\bf Forest cover influences occurrence of mammalian carnivores within Brazilian Atlantic Forest}. {\it Journal of Mammalogy} 98(6):1721-1731. \href{https://doi.org/10.1093/jmammal/gyx103}{\underline{10.1093/jmammal/gyx103}}

\item Sobral-Souza T, {\bf Vancine MH}, Ribeiro MC, Lima-Ribeiro MS. 2018. {\bf Efficiency of protected areas in Amazon and Atlantic Forest conservation: A spatio-temporal view}. {\it Acta Ecologica} 87:1-7. \href{https://doi.org/10.1016/j.actao.2018.01.001}{\underline{10.1016/j.actao.2018.01.001}}

\item Muylaert RL, {\bf Vancine MH}, Bernardo R, Oshima JEF, Sobral-Souza T, Tonetti VR, Ribeiro MC. 2018. {\bf Uma nota sobre os limites territoriais da Mata Atlântica}. {\it Oecologia Australis} 22(3):302-311. \href{https://doi.org/10.4257/oeco.2018.2203.09}{\underline{10.4257/oeco.2018.2203.09}}

\item {\bf Vancine MH}, Duarte KS, Souza YS, Giovanelli JGR, Sobrinho PMM, López A, Bovo RP, Maffei F, Lion MB, Ribeiro-Júnior JW, Brassaloti R, Ortiz C, Sawakuchi HO, Forti LR, Cacciali P, Bertoluci J, Haddad CFB, Ribeiro MC. 2018. {\bf ATLANTIC AMPHIBIANS: a data set of amphibian communities from the Atlantic Forests of South America}. {\it Ecology} 99(7):1692-1692. \href{https://doi.org/10.1002/ecy.2392}{\underline{10.1002/ecy.2392}}

\item Ferro e Silva AM, Sobral-Souza T, {\bf Vancine MH}, Muylaert RL, Abreu AP, Pelloso SM, Carvalho MDB, Andrade L, Ribeiro MC, Toledo MJ. 2018. {\bf Spatial prediction of risk areas for vector transmission of \textbf{\textit{Trypanosoma cruzi}} in the State of Paraná, southern Brazil}. {\it PLoS Neglected Tropical Diseases} 12(10):e0006907. \href{https://doi.org/10.1371/journal.pntd.0006907}{\underline{10.1371/journal.pntd.0006907}}

\item Bertassoni A, Costa RT, Gouvea JA, Bianchi RDC, Ribeiro JW, {\bf Vancine MH}, Ribeiro MC. 2019. {\bf Land-use changes and the expansion of biofuel crops threaten the giant anteater in southeastern Brazil}. {\it Journal of Mammalogy} 100(2):435-444. \href{https://doi.org/10.1093/jmammal/gyz042}{\underline{10.1093/jmammal/gyz042}}

\item Moraes AM, {\bf Vancine MH}, Moraes AM, Oliveira Cordeiro CL, Pinto MP, Lima AA, Sobral-Souza T. 2019. {\bf Predicting the potential hybridization zones between native and invasive marmosets within Neotropical biodiversity hotspots}. {\it Global Ecology and Conservation} 20:e00706.06. \href{https://doi.org/10.1016/j.gecco.2019.e00706}{\underline{10.1016/j.gecco.2019.e00706}}

\item Santos JP, Sobral‐Souza T, Brown Jr KS, {\bf Vancine MH}, Ribeiro MC, Freitas AV. 2020. {\bf Effects of landscape modification on species richness patterns of fruit‐feeding butterflies in Brazilian Atlantic Forest}. {\it Diversity and Distributions} 26(2):196-208. \href{https://doi.org/10.1111/ddi.13007}{\underline{10.1111/ddi.13007}}

\item Marjakangas E, Abrego N, Grøtan V, Lima RAF, Bello C, Bovendorp RS, Culot L, Hasui E, Muylaert RL, Lima F, Niebuhr B, Oliveira AA, Pereira L, Prado I, Stevens RD, {\bf Vancine MH}, Ribeiro MC, Galetti M, Ovaskainen O. 2020. {\bf Fragmented tropical forests lose mutualistic plant-animal interactions}. {\it Diversity and Distributions} 26(2):154-168. \href{https://doi.org/10.1111/ddi.13010}{\underline{10.1111/ddi.13010}}

\item Bello C, Cintra ALP, Barreto E, {\bf Vancine MH}, Sobral-Souza T, Graham CH, Galetti M. 2021. {\bf Environmental niche and functional role siimlarity between invasive and native palms in the Atlantic Forest}. {\it Biological Invasions} 23(3):741-754. \href{https://doi.org/10.1007/s10530-020-02400-8}{\underline{10.1007/s10530-020-02400-8}}

\item Gusmão RA, Hernandes FA, {\bf Vancine MH}, Naka LN, Doña J, Gonçalves‐Souza T. 2021. {\bf Host diversity outperforms climate as a global driver of symbiont diversity in the bird‐feather mite system}. {\it Diversity and Distributions} 27(3):416-426. \href{https://doi.org/10.1111/ddi.13201}{\underline{10.1111/ddi.13201}}

\item Da Silveira NS, {\bf Vancine MH}, Jahn AE, Pizo MA, Sobral-Souza T. 2021. {\bf Future climate change will impact the size and location of breeding and wintering areas of migratory thrushes in South America}. {\it The Condor} 123(2):duab006. \href{https://doi.org/10.1093/ornithapp/duab006}{\underline{10.1093/ornithapp/duab006}}

\item Jover A, Cabrera A, Ramos A, {\bf Vancine MH}, Suárez AM, Machell J, Pérez-Lloréns JL. 2021. {\bf Distribution of macroalgae epiphytes and host species from the Cuban marine shelf inferred from ecological modelling}. {\it Aquatic Botany} 172:103395. \href{https://doi.org/10.1016/j.aquabot.2021.103395}{\underline{10.1016/j.aquabot.2021.103395}}

\item Bercê W, Bello C, Mendes CP, {\bf Vancine MH}, Galetti M, Ballari SA. 2021. {\bf Invasive wild boar’s distribution overlap with threatened native ungulate in Patagonia}. {\it Journal of Mammalogy} 102(5):1298-1308. \href{https://doi.org/10.1093/jmammal/gyab099}{\underline{10.1093/jmammal/gyab099}}

\item Oshima JEF, Jorge MLS, Sobral-Souza T, Börger L, Keuroghlian A, Peres CA, {\bf Vancine MH}, Colleni B, Ribeiro MC. 2021. {\bf Setting priority conservation management regions to reverse rapid range decline of a key neotropical forest ungulate}. {\it Global Ecology and Conservation} 31:e01796. \href{https://doi.org/10.1016/j.gecco.2021.e01796}{\underline{10.1016/j.gecco.2021.e01796}}

\item Monteiro ECS, Pizo MA, {\bf Vancine MH}, Ribeiro MC. 2022. {\bf Forest cover and connectivity have pervasive effects on the maintenance of evolutionary distinct interactions in seed dispersal networks}. {\it Oikos} 2022(2):oik.08240. \href{https://doi.org/10.1111/oik.08240}{\underline{10.1111/oik.08240}}

\item Ribeiro-Souza P, Graipel ME, Astúa D, {\bf Vancine MH}, Pires JSR. 2022. {\bf Effects of climate change on distribution and areas that protect two neotropical marsupials associated with aquatic environments}. {\it Ecological Informatics} 68:101570. \href{https://doi.org/10.1016/j.ecoinf.2022.101570}{\underline{10.1016/j.ecoinf.2022.101570}}

\item Costa MF, Francisconi AF, {\bf Vancine MH}, Zucchi MI. 2022. {\bf Climate change impacts on the \textbf{\textit{ Copernicia alba}} and \textbf{\textit{Copernicia prunifera}} (Arecaceae) distribution in South America}. {\it Brazilian Journal of Botany} 45:807-818. \href{https://doi.org/10.1007/s40415-022-00801-8}{\underline{10.1007/s40415-022-00801-8}}

\item Muylaert RL, Kingston T, Luo J, {\bf Vancine MH}, Galli N, Carlson CJ, John RS, Rulli MC, Hayman DT. 2022. {\bf Present and future distribution of bat hosts of sarbecoviruses: implications for conservation and public health}. {\it Proceedings of the Royal Society B} 289(1975):20220397. \href{https://doi.org/10.1098/rspb.2022.0397}{\underline{10.1098/rspb.2022.0397}}

\item Borges GA, Mancilla G, Siqueira AB, {\bf Vancine MH}, Ribeiro MC, Maia JCS. 2022. {\bf The fate of vegetation remnants in the southern Amazon’s largest threatened hotspot: part (I) a 33-year analysis of LULCC in the Tapajos River basin, Brazil}. {\it Research, Society and Development} 11(10):e448111032553. \href{https://doi.org/10.33448/rsd-v11i10.32553}{\underline{10.33448/rsd-v11i10.32553}}

\item Galetti M, Carmignotto AP, Percequillo AR, Santos MCO, Ferraz KMPMB, Lima F, {\bf Vancine MH}, Muylaert RL, Bonfim FCG, Magioli M, Abra FD, Chiarello AG, Duarte JMB, Morato R, de Mello Beisiegel B, Olmos F, Galetti Jr. PM, Ribeiro MC. 2022. {\bf Mammals in São Paulo State: diversity, distribution, ecology, and conservation}. {\it Biota Neotropica}  22(spe):e20221363. \href{https://doi.org/10.1590/1676-0611-bn-2022-1363}{\underline{10.1590/1676-0611-bn-2022-1363}}

\item Santos PM, Ferraz KMPMB, Ribeiro MC, Niebuhr BB, {\bf Vancine MH}, Chiarello AC, Paglia AP. 2022. {\bf Natural forest regeneration on anthropized landscapes could overcome climate change effects on the endangered maned sloth (\textbf{\textit{ Bradypus torquatus}}, Illiger 1811)}. {\it Journal of Mammology} 103(6):1383-1396. \href{https://doi.org/10.1093/jmammal/gyac084}{\underline{10.1093/jmammal/gyac084}}

\item Dutra VAB, {\bf Vancine MH}, Lima AAM, Toledo PM. 2023. {\bf Dinâmica da paisagem e fragmentação de ecossistemas em três bacias hidrográficas na Amazônia Oriental entre 1985 e 2019}. {\it Revista Brasileira de Geografia Física} 16(02):936-949. \href{https://doi.org/10.26848/rbgf.v16.2.p936-949}{\underline{10.26848/rbgf.v16.2.p936-949}}

\item Amaral IS, Pereira JB, {\bf Vancine MH}, Morales AE, Althoff SL, Gregorin R, Pereira MJR, Valiali VH, Oliveira LR. 2023. {\bf Where do they live? Predictive geographic distribution of \textbf{\textit{Tadarida brasiliensis brasiliensis}} (Chiroptera, Molossidae) in South America}. {\it Neotropical Biology and Conservation} 18(3):139-156. \href{https://doi.org/10.3897/neotropical.18.e101390}{\underline{10.3897/neotropical.18.e101390}}

\item Anunciação PA, Ernst R, Martello F, {\bf Vancine MH}, Ribeiro MC, Carvalho LMT. 2023. {\bf Climate-driven loss of taxonomic and functional richness in Atlantic Forest anurans}. {\it Perspectives in Ecology and Conservation} 21(4):274-285. \href{https://doi.org/10.1016/j.pecon.2023.09.001}{\underline{10.1016/j.pecon.2023.09.001}}

\item Teixeira JVS, Bonfim FCG, {\bf Vancine MH}, Ribeiro MC, Oliveira LC. {\bf Effect of landscape attributes at multiple scales on the occurrence of the threatened golden-headed lion tamarin, \textbf{\textit{Leontopithecus chrysomelas}} Kuhl, 1820 (Primates, Callitrichidae)}. 2023. {\it American Journal of Primatology} 86(4):e23588. \href{https://doi.org/10.1002/ajp.23588}{\underline{10.1002/ajp.23588}}

\item Tonetti V, Bocalini F, Schunck F, {\bf Vancine MH}, Butti M, Ribeiro MC, Pizo M, Balmford A. 2024. {\bf The Protected Areas network may be inefficient to cover biodiversity in a fragmented tropical hotspot under different climate scenarios}. {\it Perspectives in Ecology and Conservation} 22(1):63-71. \href{https://doi.org/10.1016/j.pecon.2023.12.002}{\underline{10.1016/j.pecon.2023.12.002}}

\item {\bf Vancine MH}, Muylaert RL, Niebuhr BB, Oshima JEF, Tonetti V, Bernardo R, De Angelo C, Rosa MR, Grohmann CH, Ribeiro MC. 2024. {\bf The Atlantic Forest of South America: spatiotemporal dynamics of remaining vegetation and implications for conservation}. {\it Biological Conservation} 291:110499. \href{https://doi.org/10.1016/j.biocon.2024.110499}{\underline{10.1016/j.biocon.2024.110499}}

\item Barbosa FS, {\bf Vancine MH}, Ribeiro MC, Siminski A. {\bf Análise espacial da fragmentação da paisagem: um estudo de caso sobre os fragmentos florestais destinados à Reserva Legal em duas bacias hidrográficas no Estado do Piauí}. {\it Revista Observatório de la Economía Latinoamericana} 22(11):e7996. \href{https://doi.org/10.55905/oelv22n11-247}{\underline{10.55905/oelv22n11-247}}

\item Marques RCM, {\bf Vancine MH}, Súarez YR, Pereira JG, Domingos JD, Silva ABB, Pereira, ZV. Dinâmica espaço temporal: variações na composição e configuração da vegetação. {\it Revista Brasileira de Geografia Física} \it{in press.}

\end{enumerate} 

{\bf Books}

\begin{enumerate} 
\item Da Silva FR, Gonçalves-Souza T, Paterno GB, Provete DB, {\bf Vancine MH}. 2022. {\bf Análises Ecológicas no R}. Nupeea: Recife, PE, Canal 6: São Paulo. 640 p. ISBN 978-85-7917-564-0. \href{https://analises-ecologicas.com/}{\underline{link}}
\end{enumerate} 

{\bf For all publications, citations, and journal referee}
\\\href{https://orcid.org/0000-0001-9650-7575}{\underline{ORCID}}, \href{https://www.webofscience.com/wos/author/record/837504}{\underline{Web of Science}}, \href{https://www.scopus.com/authid/detail.uri?authorId=57193451888}{\underline{Scopus}}, \href{https://scholar.google.com/citations?user=i-2xZBQAAAAJ}{\underline{Google Scholar}}

\end{rSection}

%-------------------------------------------------------------------------
%	Manuscripts in review
%-------------------------------------------------------------------------

\begin{rSection}{Manuscripts in review}

\begin{enumerate} 

\item {\bf Vancine MH*}, Niebuhr BB*, Muylaert RL, Oshima JEF, Tonetti V, Bernardo R, Alves RSC, Zanette EM, Souza VC, Giovanelli JGR, Grohmann CH, Galetti M, Ribeiro MC. {\bf ATLANTIC  SPATIAL:  a  data  set  of  landscape,  topographic,  hydrologic  and  anthropogenic metrics for the Atlantic Forest}. {\it Ecology (Data Paper)}. Preprint: {\it EcoEvoRxiv}. \href{https://doi.org/10.32942/X26P58}{\underline{10.32942/X26P58}}

\item Alves-Ferreira G*, {\bf Vancine MH*}, Mota FMM, Bello C, Sobral-Souza T, Percequillo AR, Lacher Jr TE, Galetti M, Bovendorp RS. {\bf From hot to cold spots: climate change will modify endemism centers of small mammals in a biodiversity hotspot}. {\it Diversity and Distributions}.

\item Ribeiro-Souza P, Graipel ME, Oshima JEF, {\bf Vancine MH}, Lima F, Pires JSR, Ribeiro MC. {\bf The future of the wild felines under climate and landscape changes in the Atlantic Forest}. {\it Perspectives in Ecology and Conservation}.

\item Goebel LGA, {\bf Vancine MH}, Bogoni JA, Longo GR, Calicis M, Fearside PM, Palmeirim AF, Santos-Filho M. {\bf Spatio-temporal changes in forest cover and landscape configuration reveal an imminent threat in an Amazonian state at the arc of deforestation}. {\it Environmental Conservation.}

\item Carvalho T, Falconi N, White T, Anjos LA, Giasson LOM, {\bf Vancine MH}, Haddad CFB, Toledo LF, Becker CG. {\bf The Role of Seasonal Migration in Predicting Amphibian Population Persistence Across Fragmented Tropical Landscapes: An Individual-Based Model}. {\it Biodiversity and Conservation.}

\item Gonçalves-Souza T, Chase J, Haddad N, {\bf Vancine MH}, Melo FPL, Aizen M, Bernard E, Chiarello GA, Didham R, Faria D, Gibb H, Lima M, Magnago L, Mariano Neto E, Nogueira A, Nemésio A, Passamani M, Pinho BX, Rocha-Santos L, Rodrigues R, Safar N, Santos B, Soto-Werschitz A, Tabarelli M, Uehara-Prado M, Vasconcelos H, Vieira S, Sanders, N. {\bf Increasing species turnover does not alleviate biodiversity loss in fragmented landscapes}. {\it Nature.}

\item Gonçalves-Souza T*, {\bf Vancine MH*}, Sanders NJ, Haddad N, (…), Chase JM. LandFrag: a dataset to investigate the effects of habitat loss and fragmentation on biodiversity in forest fragments. {\it Global Ecology and Biogeography (Data Article)}.

\item {\bf Vancine MH}, Dodonov P, Vilela B, Diele-Viegas L, Souza VC, Nunes FAS, Silva AJO, Milz B, Kita CA, Mello MAR, Muylaert RL. The challenges and nuances of teaching the R programming language to ecologists. {\it Oecologia Australis}.

\item Marques RCM, {\bf Vancine MH}, Súarez YR, Pereira JG, Domingos JD, Barros AB, Pereira, ZV. Landscape connectivity on the taxonomic diversity of plant species in seed rain. {\it Annals of the Brazilian Academy of Sciences}.

\item Longo G, Goebel LG, {\bf Vancine MH}, Fearnside P. Brazil’s Highway BR-319: Tearing the Amazon apart. {\it Annals of the Brazilian Academy of Sciences}.

\end{enumerate} 

*Co-first author

\end{rSection}

%-------------------------------------------------------------------------
%	Teaching Experience
%-------------------------------------------------------------------------

\begin{rSection}{Teaching experience}

{\bf Graduation courses (300 h)}
\begin{itemize} 
\item {\bf Ecological Niche Modeling in R (30 h)} \hfill{\em 2016}\\
Teaching assistant, Ecology and Biodiversity Graduate Program, Graduate Program, São Paulo State University (UNESP)

\item {\bf Ecological Niche Modeling: theory and practice (45h)} \hfill{\em 2016}\\
Teaching assistant, Ecology Graduate Program, State University of Campinas (Unicamp)

\item {\bf Ecological Niche Modeling: theory and practice (60 h)} \hfill{\em 2017}\\
Teaching assistant, Ecology and Biodiversity Graduate Program, São Paulo State University (UNESP)

\item {\bf Introduction to Geoprocessing for Ethnobiology and Conservation (45 h)} \hfill{\em 2019}\\
Invited teacher, Ethnobiology and Nature Conservation Graduate Program, Rural Federal University of Pernambuco (UFRPE)

\item {\bf Introduction to the use of geospatial data in R (60 h)} \hfill{\em 2020}\\
Invited teacher, Ecology, Biodiversity and Evolution Graduate Program, São Paulo State University (UNESP)

\item {\bf Introduction to the use of geospatial data in R (60 h)} \hfill{\em 2021}\\
Invited teacher, Ecology, Biodiversity and Evolution Graduate Program, São Paulo State University (UNESP)
\end{itemize}

{\bf Undergraduate courses (102 h)}
\begin{itemize}
\item {\bf Quantitative Ecology (60 h)} \hfill{\em 2015}\\
Teaching assistant, Ecology Undergraduate, São Paulo State University (UNESP)

\item {\bf Statistical Models in Ecology (12 h)} \hfill{\em 2018}\\
Teaching assistant, Ecology Undergraduate, São Paulo State University (UNESP)

\item {\bf Statistical Models in Ecology (30 h)} \hfill{\em 2020}\\
Teacher, Ecology Undergraduate, São Paulo State University (UNESP)

\end{itemize}

{\bf Workshops (112 h)}
\begin{itemize}
\item {\bf Organization of data in electronic sheets - Calc (08 h)} \hfill{\em 2014}\\
Workshop instructor, XXV Ecology Studies Week, São Paulo State University (UNESP) 

\item {\bf Introduction to software R (08 h)} \hfill{\em 2015}\\
Workshop instructor, XXVI Ecology Studies Week, São Paulo State University (UNESP) 

\item {\bf Field herpetology (15 h)} \hfill{\em 2016}\\
Workshop instructor, XXVII Ecology Studies Week, São Paulo State University (UNESP) 

\item {\bf Introduction to software R (16 h)} \hfill{\em 2016}\\
Workshop instructor, XXVII Ecology Studies Week, São Paulo State University (UNESP) 

\item {\bf Introduction to Species Distribution Modeling using R Language: theory and practice (07 h)} \hfill{\em 2018}\\
Workshop instructor, 9º Brazilian Congress of Herpetology, State University of Campinas (Unicamp)

\item {\bf Field work with amphibians (08 h)} \hfill{\em 2018}\\
Workshop instructor, 29ª Biology Studies Week, São Paulo State University (UNESP) 

\item {\bf Introduction to R Language (08 h)} \hfill{\em 2019}\\
Workshop instructor, XXX Ecology Studies Week, São Paulo State University (UNESP) 

\item {\bf Introduction to R Language (08 h)} \hfill{\em 2019}\\
Workshop instructor, 30ª Biology Studies Week, São Paulo State University (UNESP)

\item {\bf Geoprocessing with QGIS (03 h)} \hfill{\em 2021}\\
Workshop instructor, 32ª Biology Studies Week, São Paulo State University (UNESP)

\item {\bf Introduction to species distribution modeling using the R language (06 h)} \hfill{\em 2021}\\
Workshop instructor, Mastozóologos Organizados em uma Conferência Online (MOCÓ), Sociedade Brasileira de Mastozoologia

\item {\bf Introduction to the R language for data manipulation and visualization (08 h)} \hfill{\em 2021}\\
Workshop instructor, XXXI Ecology Studies Week, São Paulo State University (UNESP)

\item {\bf Introduction to the R language for data manipulation and visualization (09 h)} \hfill{\em 2022}\\
Workshop instructor, XXXIII Ecology Studies Week, São Paulo State University (UNESP)

\item {\bf Analyzing spatial patterns in biodiversity data (06 h)} \hfill{\em 2024}\\
Workshop instructor, São Paulo School of Advanced Science (SPSAS), State University of Campinas (Unicamp)

\item {\bf Introduction to landscape metrics in R (02 h)} \hfill{\em 2024}\\
Workshop instructor, I Brazilian Symposium on Ecological Synthesis, Federal University of Santa Maria (UFSM)
\end{itemize}

{\bf Guest lectures (10 h)}
\begin{itemize}
\item {\bf Introduction to the R language (02 h)} \hfill{\em 2023}\\
Teacher, Statistical Models in Ecology, São Paulo State University (UNESP) 

\item {\bf Introduction to Numerical Ecology (02 h)} \hfill{\em 2023}\\
Teacher, Quantitative Ecology, São Paulo State University (UNESP)

\item {\bf Data manipulation and programming in R (06 h)} \hfill{\em 2024}\\
Teacher, NT265/NE441 - Data analysis and graphical presentation using R language, State University of Campinas (Unicamp)
\end{itemize}
\end{rSection}

%-------------------------------------------------------------------------
%	Administrative experience
%-------------------------------------------------------------------------

\begin{rSection}{Administrative experience}
\begin{itemize}
\item {\bf Student representative} \hfill{\em 2020-2021}\\
Graduate Program in Ecology, Evolution and Biodiversity, São Paulo State University (UNESP)
\end{itemize}
\end{rSection}

%-------------------------------------------------------------------------
%   Professional affiliations and memberships
%-------------------------------------------------------------------------

\begin{rSection}{Professional affiliations and memberships}
\begin{itemize}
\item Associação Brasileira de Ecólogos (ABE) \hfill{\em 2018-Currently}
\item Associação Brasileira de Ciência Ecológica e Conservação (ABECO) \hfill{\em 2021-Currently}
\end{itemize}
\end{rSection}

%-------------------------------------------------------------------------
%   Student supervision
%-------------------------------------------------------------------------

\begin{rSection}{Student supervision}
\begin{itemize}
\item {\bf Helena Thereza Carvalho de Oliveira} \hfill{\em 2021}\\
{\it Distribuição do padrão reprodutivo em comunidades de anuros na Mata Atlântica.} \href{https://repositorio.unesp.br/items/c8b867f6-4fd0-4e07-853a-abbf74d64c23}{\underline{link}}\\
Advisor, Undergraduate thesis, Biological Sscience, São Paulo State University (UNESP)

\item {\bf Bruno Eduardo Ribeiro Silva} \hfill{\em 2022}\\
{\it Comparação da distribuição de espécies inferidos por Modelos de Nicho Ecológicos e mapa de especialistas da IUCN para anfíbios anuros da América do Sul.} \href{https://repositorio.unesp.br/items/190f34f7-f0a7-4458-9dfc-ef8890bf0ca2}{\underline{link}}\\
Advisor, Undergraduate thesis, Ecology, São Paulo State University (UNESP)

\item {\bf Maria Eduarda Furlan} \hfill{\em 2025}\\
{\it Inferindo a circulação do vírus da raiva e risco zoonótico com aplicações para a saúde pública.}\\
Advisor, Undergraduate thesis, Ecology, São Paulo State University (UNESP)

\item {\bf Mariana Trindade Lahr} \hfill{\em 2025}\\
{\it Movimento e mortalidade: efeito das rodovias sobre a fauna e conservação ambiental no Brasil.}\\
Advisor, Undergraduate thesis, Ecology, São Paulo State University (UNESP)
\end{itemize}
\end{rSection}

%-------------------------------------------------------------------------
%   Meetings and presentations
%-------------------------------------------------------------------------

\begin{rSection}{Meetings and presentations}

{\bf Latest conferences}
\begin{itemize}
\item {\bf Brazilian Amphibian Conservation Symposium (ANFoCO)} \hfill{\em 2018}
\item {\bf Brazilian Congress of Herpetology} \hfill{\em 2019}
\item {\bf II Workshop of Community Ecology} \hfill{\em 2020}
\item {\bf Mastozoologists Organized in an Online Conference} \hfill{\em 2021}
\item {\bf FOSS4G (Free and Open Source Software for Geospatial)} \hfill{\em 2021}
\item {\bf I Brazilian Symposium on Ecological Synthesis} \hfill{\em 2024}
\end{itemize}

\end{rSection}

%-------------------------------------------------------------------------
%   Services
%-------------------------------------------------------------------------

\begin{rSection}{Services}
\begin{itemize}
\item Journal referee:\\
Zoologia, PLOS One, Biological Invasions, Hydrobiologia, Scientific Data, Diversity and Distributions, Papéis Avulsos de Zoologia, Conservation Biology, Austral Ecology, Scientia Plena, Ecosystem Services, PeerJ, Global Change Biology, Journal of Biogeography, Biotropica, Ocean and Coastal Management, Journal of Applied Ecology 
\end{itemize}
\end{rSection}

%-------------------------------------------------------------------------
%	Skills
%-------------------------------------------------------------------------

\begin{rSection}{Skills}

{\bf Programming languages}
\begin{itemize}
\item R (advanced), tidyverse (advanced), markdown (advanced), quarto (intermediate), shiny (basic), git (intermediate), Python (basic), Julia (basic), JavaScript (basic), shell/bash (basic), HTML/CSS (basic), LaTeX (intermediate)
\end{itemize}

{\bf R packages}
\begin{itemize} 
\item {\it lsmetrics}: multiscale calculation and spatialization of landscape metrics using GRASS GIS\\
Authors: Bernardo B. Niebuhr*, {\bf Maurício Humberto Vancine*}, Felipe Martello Ribeiro, Renata L. Muylaert, John Wesley Ribeiro, Milton Cezar Ribeiro\\
Link: \href{https://mauriciovancine.github.io/lsmetrics}{\underline{mauriciovancine.github.io/lsmetrics}}

\item {\it atlanticr}: provides data from the ATLANTIC, data papers from Atlantic Forest\\
Authors: {\bf Maurício Humberto Vancine}, Bernardo Niebuhr, Renata L. Muylaert, Mauro Galetti, Milton Cezar Ribeiro\\
Link: \href{https://mauriciovancine.github.io/atlanticr}{\underline{mauriciovancine.github.io/atlanticr}}

\item {\it amphiBR}: dataset from the official publication of the List of Amphibians in Brazil published by the Brazilian Society of Herpetology\\
Authors: Paulo Barros de Abreu Junior, {\bf Maurício Humberto Vancine}, Diogo B. Provete\\
Link: \href{https://paulobarros.github.io/amphiBR}{\underline{paulobarros.github.io/amphiBR}}

\item {\it ecodados}: ecological data base for teaching statistics\\
Authors: Gustavo Paterno, Diogo B. Provete, Fernando Rodrigues da Silva, Thiago Gonçalves-Souza, {\bf Maurício Humberto Vancine}\\
Link: \href{https://paternogbc.github.io/ecodados/}{\underline{paternogbc.github.io/ecodados}}

\item {\it lsma}: landscape-structure multiscale analysis using R\\
Authors: Wilson Frantine-Silva, Lazaro S. Carneiro, André Luis Regolin, {\bf Maurício Humberto Vancine}, Juliana S. Santos, Bernardo Brandão Niebuhr, Edson Valmorbida, Maria Cristina Gaglianone, Milton C. Ribeiro\\
Link: \href{https://github.com/wilsonfrantine/landscapeDecoupler}{\underline{wilsonfrantine/lsma}}

\end{itemize} 

*Co-first author

{\bf Statistical knowledge}

\begin{itemize} 
\item {\it Descriptive statistics}: tabular data manipulation (advanced), spatial data manipulation (advanced), tabular data visualization (advanced), spatial data visualization (advanced).

\item {\it Inferential statistics}: frequentist (advanced), likelihood (intermediate), model selection (intermediate), bayesian (basic).
\end{itemize}

{\bf Software, applications and platforms}
\begin{itemize}
\item QGIS (advanced), GRASS GIS (advanced), GNU/Linux (intermediate), Fragstats (intermediate), GuidosToolbox (intermediate), Google Earth Engine (basic), Inkscape (basic), git (basic), Sublime Text (basic), Emacs (basic)
\end{itemize}

{\bf Languages}
\begin{itemize}
\item English (intermediate)
\item Spanish (basic)
\item Portuguese (native speaker)
\end{itemize}

\end{rSection}

%-------------------------------------------------------------------------
%   Media
%-------------------------------------------------------------------------

\begin{rSection}{Media}

{\bf News}
\begin{itemize} 
\item {\bf Pesquisadores mapeiam áreas com potencial para futuros surtos de doenças semelhantes à Covid-19}. {\it Jornal da UNESP}. \href{https://jornal.unesp.br/2022/06/23/pesquisadores-mapeiam-areas-com-potencial-para-futuros-surtos-de-doencas-semelhantes-a-covid-19/}{\underline{link}} \hfill{\em 2022}
\item {\bf Alto Tietê somou mais de 30 hectares desmatados em quatro anos}. {\it G1}. \href{https://g1.globo.com/sp/mogi-das-cruzes-suzano/noticia/2023/06/05/em-quatro-anos-alto-tiete-somou-mais-de-30-hectares-desmatados.ghtml}{\underline{link}} \hfill{\em 2023}
\item {\bf Novo estudo destaca preocupação com estado de conservação da Mata Atlântica}. {\it Um Só Planeta}. \href{https://umsoplaneta.globo.com/biodiversidade/noticia/2024/06/24/novo-estudo-destaca-preocupacao-com-estado-de-conservacao-da-mata-atlantica.ghtml?utm_campaign=na_midia__n_23_junho_2024&utm_medium=email&utm_source=RD+Station}{\underline{link}} \hfill{\em 2024}
\item {\bf Novo estudo destaca preocupação com estado de conservação da Mata Atlântica}. {\it Folha de São Paulo}. \href{https://www1.folha.uol.com.br/folha-social-mais/2024/06/novo-estudo-destaca-preocupacao-com-estado-de-conservacao-da-mata-atlantica.shtml?utm_campaign=na_midia__n_23_junho_2024&utm_medium=email&utm_source=RD+Station}{\underline{link}} \hfill{\em 2024}
\item {\bf Novo estudo destaca preocupação com estado de conservação da Mata Atlântica}. {\it Agência FAPESP}. \href{https://agencia.fapesp.br/novo-estudo-destaca-preocupacao-com-estado-de-conservacao-da-mata-atlantica/52013}{\underline{link}} \hfill{\em 2024}
\item {\bf Novo estudo destaca preocupação com estado de conservação da Mata Atlântica}. {\it Jornal da UNESP}. \href{https://jornal.unesp.br/2024/03/15/novo-estudo-destaca-preocupacao-com-estado-de-conservacao-da-mata-atlantica/}{\underline{link}} \hfill{\em 2024}
\item {\bf Condomínios, estradas e caça: Mata Atlântica enfrenta novos riscos}. {\it Olhar Digital [Ciência e Espaço]}. \href{https://olhardigital.com.br/2024/06/25/ciencia-e-espaco/condominios-estradas-e-caca-mata-atlantica-enfrenta-novos-riscos/}{\underline{link}} \hfill{\em 2024}
\item {\bf Desmatamento desacelera na Mata Atlântica, mas fragmentação aumenta}. {\it Mongabay Brasil}. \href{https://brasil.mongabay.com/2024/07/desmatamento-desacelera-na-mata-atlantica-mas-fragmentacao-aumenta}{\underline{link}} \hfill{\em 2024}
\item {\bf Unidades de Conservação: guardiãs da Mata Atlântica}. {\it Ekos Brasil}. \href{https://www.ekosbrasil.org/unidades-de-conservacao-guardias-da-mata-atlantica}{\underline{link}} \hfill{\em 2024}
\item {\bf Consciência ambiental é aceleradora de projetos de restauração na Mata Atlântica}. {\it Um Só Planeta}. \href{https://umsoplaneta.globo.com/clima/noticia/2024/09/26/consciencia-ambiental-e-aceleradora-de-projetos-de-restauracao-na-mata-atlantica.ghtml}{\underline{link}} \hfill{\em 2024}
\item {\bf No Clima dos Biomas: com restauração de florestas e nascentes de água, projetos buscam recuperar uma Mata Atlântica degradada}. {\it Um Só Planeta} \href{https://umsoplaneta.globo.com/clima/noticia/2024/09/26/no-clima-dos-biomas-com-restauracao-de-florestas-e-nascentes-de-agua-projetos-buscam-recuperar-uma-mata-atlantica-degradada.ghtml}{\underline{link}} \hfill{\em 2024}
\end{itemize} 

{\bf YouTube}
\begin{itemize} 
\item {\bf Métricas de paisagem no R}. {\it GeoCast Brasil}. \href{https://youtu.be/RCTrLx_33D8}{\underline{link}} \hfill{\em 2019}
\item {\bf We R Live 01: Como começar no R}. {\it GeoCast Brasil}. \href{https://youtu.be/ZORFVdwtJ1U}{\underline{link}} \hfill{\em 2020}
\item {\bf We R Live 02: Elaborando mapas no R}. {\it GeoCast Brasil}. \href{https://youtu.be/eHht0n3Ppcg}{\underline{link}} \hfill{\em 2020}
\item {\bf We R Live 03: Elaborando mapa no R com tmap}. {\it GeoCast Brasil}. \href{https://youtu.be/BmlM25XQ3QA}{\underline{link}} \hfill{\em 2020}
\item {\bf We R Live 04: Manipulando dados raster no R}. {\it GeoCast Brasil}. \href{https://youtu.be/dFC9SuGLuX8}{\underline{link}} \hfill{\em 2020}
\item {\bf We R Live 05: Manipulando dados raster no R II}. {\it GeoCast Brasil}. \href{https://youtu.be/AKJo_Q0dsMI}{\underline{link}} \hfill{\em 2020}
\item {\bf We R Live 06: Extraindo dados climáticos para pontos}. {\it GeoCast Brasil}. \href{https://youtu.be/-_ODMFDU6ck}{\underline{link}} \hfill{\em 2020}
\item {\bf We R Live 07: Introdução à estatística Espacial I}. {\it GeoCast Brasil}. \href{https://youtu.be/fHWD4qyKj84}{\underline{link}} \hfill{\em 2020}
\item {\bf We R Live 08: Introdução à Estatística Espacial II}. {\it GeoCast Brasil}. \href{https://youtu.be/BCl_V-SpQec}{\underline{link}} \hfill{\em 2020}
\item {\bf We R Live 09: Introdução à estatística Espacial III}. {\it GeoCast Brasil}. \href{https://youtu.be/eaR7pTsQFDQ}{\underline{link}} \hfill{\em 2020}
\item {\bf We R Live 10: Join como associar dados à vetores}. {\it GeoCast Brasil}. \href{https://youtu.be/BvURukaIDM0}{\underline{link}} \hfill{\em 2020}
\item {\bf We R Live 11: florestal - Pacote R para inventário florestal}. {\it GeoCast Brasil}. \href{https://youtu.be/xOgsVywKADI}{\underline{link}} \hfill{\em 2020}
\item {\bf We R Live 13: Introdução à Estatística Espacial IV}. {\it GeoCast Brasil}. \href{https://youtu.be/IsOJvaWdyXE}{\underline{link}} \hfill{\em 2020}
\item {\bf We R Live 14: Geoestatística com Jorge Kazuo Yamamoto}. {\it GeoCast Brasil}. \href{https://youtu.be/BvURukaIDM0}{\underline{link}} \hfill{\em 2020}
\item {\bf We R Live 15: Introdução à estatística Espacial V}. {\it GeoCast Brasil}. \href{https://youtu.be/IeGTr7mjZIc}{\underline{link}} \hfill{\em 2020}
\item {\bf We R Live 16: R Markdown: usando o R para comunicar seus resultados}. {\it GeoCast Brasil}. \href{https://youtu.be/6oPZ5sGt6LA}{\underline{link}} \hfill{\em 2020}
\item {\bf Unesp realiza estudo para verificar a importância do isolamento social}. {\it TV Claret}. \href{https://youtu.be/RP6rUQpberE}{\underline{link}} \hfill{\em 2020}
\item {\bf Uso e aplicações de modelos estatísticos em Ecologia e Geografia}. {\it EstaTiDados}. \href{https://youtu.be/pUavO7dVRGk}{\underline{link}} \hfill{\em 2020}
\item {\bf Aplicações da Cartografia para a Ecologia Espacial}. {\it Ciclo de Palestras de Cartografia Básica (UFF)}. \href{https://youtu.be/csh1BPH_H8I}{\underline{link}} \hfill{\em 2021}
\item {\bf Estatística e análise de dados espaciais no R: um estudo de caso com dados do Lago Walker}. {\it GeoCast Brasil}. \href{https://youtu.be/csh1BPH_H8I}{\underline{link}} \hfill{\em 2021}
\item {\bf Conhecendo Hugo Apéro: fazendo blogs com blogdown e R}. {\it GeoCast Brasil}. \href{https://youtu.be/4Ixl2RjZEYI}{\underline{link}} \hfill{\em 2021}
\item {\bf O fascinante mundo do GRASS GIS}. {\it GeoCast Brasil}. \href{https://youtu.be/_pohWjE4eiA}{\underline{link}} \hfill{\em 2021}
\item {\bf O Fascinante Mundo do GRASS}. {\it Fascinante Mundo do Sensoriamento Remoto}. \href{https://youtu.be/vp6frd89y9E}{\underline{link}} \hfill{\em 2021}
\item {\bf Modelos em Ecologia: extrapolando nosso conhecimento sobre a distribuição das espécies}. {\it DEA UFV}. \href{https://youtu.be/Kcc-eIhqtlE}{\underline{link}} \hfill{\em 2021}
\item {\bf Análises Ecológicas no R (Lançamento)}. {\it Análises Ecológicas no R}. \href{https://youtu.be/jYrneB95nes}{\underline{link}} \hfill{\em 2022}
\item {\bf Análise de dados geoespaciais no R}. {\it Grupo de Estudos em Ecologia Quantitativa (GEEQ)}. \href{https://youtu.be/qySBvCu7IS0}{\underline{link}} \hfill{\em 2024}	
\end{itemize} 

{\bf Podcasts and interviews}
\begin{itemize} 
\item {\bf GeoCast Brasil}. {\it O Fascinante Mundo do Sensoriamento Remoto}. \href{https://open.spotify.com/episode/0Kn5k9039vM8bZgb4YdlJJ?si=S5hfhNc6S6GfDFaJSsaQDA}{\underline{link}} \hfill{\em 2020}
\item {\bf Conversa com Ecólogo II - A importância da análise de dados}. {\it Associação Brasileira de Ecólogos (ABE) - Conversa com Ecólogo}. \href{https://www.instagram.com/p/CFQVjMtH6qk/}{\underline{link}} \hfill{\em 2020}
\item {\bf Você tem medo de errar?} {\it DesAbraçando Árvores}. \href{https://www.desabrace.com.br/092-voce-tem-medo-de-errar/}{\underline{link}} \hfill{\em 2022} 
\end{itemize} 
\item {\bf Pesquisa da Unesp aponta melhora na conservação da Mata Atlântica}. {\it RecordTV Interior SP}. \href{https://recordtv.r7.com/recordtv-interior-sp/sp-record/pesquisa-da-unesp-aponta-melhora-na-conservacao-da-mata-atlantica-27032024?utm_source=link_direto&utm_medium=share-bar&utm_campaign=r7-topo}{\underline{link}} \hfill{\em 2024}
\end{rSection}

%-------------------------------------------------------------------------
%   References
%-------------------------------------------------------------------------

\begin{rSection}{References}
\begin{itemize}
\item {\bf Dr. Milton C. Ribeiro, São Paulo State University (UNESP), Brazil}\\
milton.c.ribeiro@unesp.br, +55-19-3526-9680
\item {\bf Dr. Célio F. B. Haddad, São Paulo State University (UNESP), Brazil}\\
haddad1000@gmail.com, +55-19-3526-4302
\item {\bf Dr. Mathias Pires Mistretta, State University of Campinas (Unicamp), Brazil}\\
mathiasmpires@gmail.com, +55-19-3521-6275
\item {\bf Dr. Thiago Gonçalves-Souza, University of Michigan, USA}\\
tgoncalv@umich.edu, +1-734-596-0508
\item {\bf Dr. Renata L. Muylaert, Massey University, New Zealand}\\
R.deLaraMuylaert@massey.ac.nz, +64-06-356-9099 ext. 85217
\item {\bf Dr. Bernardo B. Niebuhr, Norwegian Institute for Nature Research (NINA), Norway}\\
bernardo.brandao@nina.no, +47-406-41-783
\item {\bf Dr. João G. R. Giovanelli, Seleção Natural, Brazil}\\
joao@selecaonatural.net, +47-19-991-037-348
\end{itemize}
\end{rSection}

\end{document}----------------------------