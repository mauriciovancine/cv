\documentclass{resume}

\usepackage[left=0.75in,top=0.6in,right=0.75in,bottom=0.75in]{geometry}
\usepackage[utf8]{inputenc}
\usepackage[T1]{fontenc}

% opções de fonte
\usepackage[sc]{mathpazo}
\linespread{1.05}  % palladio precisa de mais espaçamento entre linhas

\name{Maurício H. Vancine}
\address{Universidade Estadual de Campinas (Unicamp), Campinas, SP, Brasil}
\address{
\href{mailto:}{mauricio.vancine@gmail.com} 
$\ \bullet\ $ 
\href{https://github.com/mauriciovancine}{GitHub} 
$\ \bullet\ $ 
\href{https://www.linkedin.com/in/mauricio-vancine/}{LinkedIn} 
$\ \bullet\ $ 
\href{https://mauriciovancine.github.io}{Website}}
\address{Ecólogo, Cientista de Dados e Doutor em Ecologia}

% cor dos links
\usepackage{xcolor} % Pacote para manipulação de cores
\usepackage{hyperref} % Pacote para links

\hypersetup{
    colorlinks=true, % Ativa a coloração dos links
    linkcolor=blue, % Cor dos links internos
    urlcolor=blue,  % Cor dos links externos (URLs)
    citecolor=blue  % Cor das citações
}

\begin{document}

%------------------------------------------------------------------------
%   Resumo de carreira
%------------------------------------------------------------------------

\begin{rSection}{Resumo de carreira}
Sou Bacharel em Ecologia (2014), Mestre em Zoologia (2018) e Doutor em Ecologia (2024), todos pela Universidade Estadual Paulista, Rio Claro/SP, Brasil. Realizei mais de 700 horas de formação complementar em instituições brasileiras e internacionais, com foco em modelagem estatística e ecológica. 

Minha expertise abrange ecologia espacial, ecologia de paisagem, modelagem ecológica, modelagem de distribuição de espécies (SDM), análises ecológicas e espaciais, impactos da perda e fragmentação de habitat sobre a biodiversidade, ecologia de anfíbios e ensino da linguagem R. 

Desde 2017, publiquei mais de 35 artigos em periódicos de destaque em Ecologia. Essas publicações apresentam principalmente os resultados das minhas pesquisas sobre os efeitos da estrutura da paisagem e das mudanças climáticas na biodiversidade da América do Sul, com foco no hotspot Mata Atlântica. 

Atualmente, sou pesquisador de pós-doutorado no \href{https://www.mathiasmpires.net.br/index.html}{\underline{lab.exe}} da \href{https://unicamp.br/}{\underline{Unicamp}}, onde integro métricas de paisagem e modelagem de distribuição de espécies (SDM) para avaliar os impactos dos efeitos de borda, incêndios e mudanças climáticas sobre a distribuição e interações de animais e plantas em múltiplas escalas espaciais na Amazônia. 

Estou envolvido ativamente no desenvolvimento de ferramentas para calcular métricas de paisagem em grandes extensões espaciais utilizando a linguagem R e o GRASS GIS. Além disso, dediquei mais de 500 horas ao ensino e à promoção do uso da linguagem R por meio de workshops, disciplinas de graduação e pós-graduação, e da publicação do livro 'Análises Ecológicas no R'.
\end{rSection}

%------------------------------------------------------------------------
%   Áreas de expertise acadêmica
%------------------------------------------------------------------------

\begin{rSection}{Áreas de expertise acadêmica}
Ecologia espacial, ecologia de paisagem, modelagem ecológica, modelagem de distribuição de espécies (SDM), geoprocessamento, anfíbios, conservação, ensino, linguagem R.
\end{rSection}

%------------------------------------------------------------------------
%   Interesses de pesquisa
%------------------------------------------------------------------------

\begin{rSection}{Interesses de pesquisa}
Perda e fragmentação de habitat, métricas de paisagem, modelagem de distribuição de espécies (SDM), ecologia de comunidades, ecologia de anfíbios, biologia da conservação, ensino da linguagem R e estatística.
\end{rSection}

%------------------------------------------------------------------------
%   Educação
%------------------------------------------------------------------------

\begin{rSection}{Educação}

{\bf Universidade Estadual Paulista (UNESP)} \hfill {\em Mar 2011 - Mar 2015}\\
Bacharelado em Ecologia\hfill {\em Coeficiente de Rendimento: 3,43}\\
{\it Tese}: Efeito da fragmentação na persistência de anfíbios anuros (Amphibia: Anura) na Mata Atlântica\\
{\it Biblioteca}: \href{http://hdl.handle.net/11449/138991}{\underline{link}}\\
{\it Orientador}: Prof. Dr. Milton Cezar Ribeiro

{\bf Universidade Estadual Paulista (UNESP)} \hfill {\em Mar 2016 - Jul 2018}\\
Mestrado, Programa de Pós-Graduação em Ciências Biológicas (Zoologia)\hfill {\em Coeficiente: 4,00}\\
{\it Dissertação}: Diversidade, distribuição e efeito das mudanças climáticas nas comunidades de anfíbios da Mata Atlântica\\
{\it Biblioteca}: \href{http://hdl.handle.net/11449/154993}{\underline{link}}\\
{\it Orientador}: Prof. Dr. Célio Fernando Baptista Haddad

{\bf Universidade Estadual Paulista (UNESP)} \hfill {\em Mar 2020 - Jul 2024}\\
Doutorado, Programa de Pós-Graduação em Ecologia, Evolução e Biodiversidade \hfill {\em Coeficiente: 4,00}\\
{\it Tese}: Estrutura da paisagem como preditora da diversidade taxonômica e funcional de anfíbios na Mata Atlântica\\
{\it Biblioteca}: \href{https://hdl.handle.net/11449/256726}{\underline{link}}\\
{\it Orientador}: Prof. Dr. Milton Cezar Ribeiro

\end{rSection}

%------------------------------------------------------------------------
%   Formação complementar
%------------------------------------------------------------------------

\begin{rSection}{Formação complementar}

\begin{itemize} 

\item {\bf Análise de dados ecológicos com R (40 h)} \hfill {\em 2011}\\
Universidade Estadual Paulista (UNESP), Rio Claro, SP, Brasil

\item {\bf Biologia e conservação de anfíbios e répteis (44 h)} \hfill {\em 2015}\\
Instituto Boitatá, IBEC, Alfenas, MG, Brasil, \href{https://www.institutoboitata.org/}{\underline{link}}

\item {\bf V Southern-Summer School on Mathematical Biology (53 h)} \hfill {\em 2016}\\
Universidade Estadual Paulista (UNESP), São Paulo, SP, Brasil, \href{https://www.ictp-saifr.org/v-southern-summer-school-on-mathematical-biology}{\underline{link}} 

\item {\bf Geoprocessamento com GRASS GIS (24 h)} \hfill {\em 2016}\\
Universidade Estadual Paulista (UNESP), Rio Claro, SP, Brasil

\item {\bf Introdução à Modelagem Hierárquica (45 h)} \hfill {\em 2019}\\
Universidade Federal do Rio Grande do Sul (UFRGS), Porto Alegre, RS, Brasil, \href{http://ferrazlab.org/graduate}{\underline{link}}

\item {\bf School on Community Ecology: from patterns to principles (60 h)} \hfill {\em 2020}\\
Universidade Estadual Paulista (UNESP), São Paulo, SP, Brasil, \href{https://www.ictp-saifr.org/community-ecology-from-patterns-to-principles}{\underline{link}}

\item {\bf Hierarchical Modelling of Species Communities with the R-package Hmsc (25 h)} \hfill {\em 2020}\\
Universidade de Helsinque (Online), Helsinque, Finlândia, \href{https://www.helsinki.fi/en/researchgroups/statistical-ecology/hmsc}{\underline{link}}

\item {\bf Joint Species Distribution Modelling with HMSC (45 h)} \hfill {\em 2022}\\
Universidade de Jyväskylä (Online), Jyväskylä, Finlândia, \href{https://www.helsinki.fi/en/researchgroups/statistical-ecology/software/hmsc}{\underline{link}}

\item {\bf Jornada em Ciência de Dados (330 h)} \hfill {\em 2022}\\
Ômega Data Science (Online), Curitiba, PR, \href{https://escola.omegadatascience.com.br}{\underline{link}}

\item {\bf Codificando Soluções - Curso de Google Earth Engine para análises geoespaciais (50 h)} \hfill {\em 2025}\\
AmbGEO (Online), Florianópolis, SC, \href{https://ambgeo.com/}{\underline{link}}

\item {\bf Codificando Soluções - Curso de Python para análises geoespaciais (50 h)} \hfill {\em 2025}\\
AmbGEO (Online), Florianópolis, SC, \href{https://ambgeo.com/}{\underline{link}}
\end{itemize} 

\end{rSection}

%------------------------------------------------------------------------
%   Experiência profissional
%------------------------------------------------------------------------

\begin{rSection}{Experiência profissional}
\begin{itemize}
\item {\bf Estágio (300 h)} \hfill{\em 2014}\\
{\it Função}: Estágio curricular obrigatório com foco em Modelagem de Distribuição de Espécies (SDM)\\
{\it Instituição}: Escola Superior de Agricultura Luiz de Queiroz, Universidade de São Paulo (USP), Laboratório de Ecologia, Manejo e Conservação de Fauna Silvestre (LEMaC)\\
{\it Supervisor}: Profa. Dra. Katia Maria Paschoaletto Micchi de Barros Ferraz

\item {\bf Assistente de pesquisa (2000 h)} \hfill{\em 2015-2016}\\
{\it Função}: Análises estatísticas e espaciais da biodiversidade de mamíferos e formigas no bioma Mata Atlântica\\
{\it Instituição}: Universidade Estadual Paulista (UNESP), Laboratório de Ecologia Espacial e Conservação (LEEC)\\
{\it Supervisor}: Prof. Dr. Milton Cezar Ribeiro

\item {\bf Pesquisador de pós-doutorado} \hfill{\em 2024-presente}\\
{\it Função}: Impactos de efeitos de borda, fogo e mudanças climáticas sobre a composição e diversidade funcional da vegetação em múltiplas escalas espaciais na Amazônia\\
{\it Instituição}: Universidade Estadual de Campinas (Unicamp), Laboratório de Ecologia da Extinção (lab.exe)\\
{\it Supervisor}: Prof. Dr. Mathias Pires Mistretta

\item {\bf Cientista de dados ambientais} \hfill{\em 2019-presente}\\
{\it Função}: Analista de dados ecológicos e geoespaciais, com foco em ecologia espacial (métricas de paisagem, simulação de corredores ecológicos, geoprocessamento) e modelagem ecológica (análises estatísticas e modelagem de distribuição de espécies)\\
{\it Empresa}: Seleção Natural: consultoria\\
\end{itemize}
\end{rSection}

%------------------------------------------------------------------------
%   Experiência em pesquisa
%------------------------------------------------------------------------

\begin{rSection}{Experiência em pesquisa}
\begin{itemize}
\item {\bf Iniciação científica} \hfill{\em 2013-2015}\\
{\it Título}: Efeito da fragmentação na persistência de anfíbios anuros (Amphibia: Anura) na Mata Atlântica\\
{\it Objetivo}: Avaliar a contribuição relativa de índices de paisagem (percentual de cobertura florestal, conectividade, relevo e proximidade urbana) para a persistência das espécies, usando modelagem de distribuição de espécies.

\item {\bf Pesquisa de mestrado} \hfill{\em 2016-2018}\\
{\it Título}: Efeito das modificações na paisagem e das mudanças climáticas na persistência de anfíbios na Mata Atlântica\\
{\it Objetivo}: Apresentar uma síntese dos levantamentos das comunidades de anfíbios na Mata Atlântica; investigar como a perda e fragmentação de habitat afetam a persistência dos anfíbios; e investigar como as mudanças climáticas afetam a persistência futura do gênero {\it Brachycephalus}.

\item {\bf Pesquisa de doutorado} \hfill{\em 2020-2024}\\
{\it Título}: Estrutura da paisagem como preditora da diversidade taxonômica e funcional de anfíbios na Mata Atlântica\\
{\it Objetivo}: Analisar a estrutura das paisagens em toda a Mata Atlântica; investigar como a estrutura da paisagem afeta a diversidade taxonômica e funcional das comunidades de anfíbios; e analisar a mesma questão em nível de espécies específicas, além de investigar coocorrências usando JSDMs.

\item {\bf Pesquisa de pós-doutorado} \hfill{\em 2024-presente}\\
{\it Título}: Impactos de efeitos de borda, fogo e mudanças climáticas sobre a composição e diversidade funcional da vegetação em múltiplas escalas espaciais na Amazônia\\
{\it Objetivo}: Entender como efeitos de borda, fogo e mudanças climáticas atuando em escalas locais podem influenciar os padrões de distribuição e composição da vegetação em diferentes escalas espaciais no bioma Amazônico, afetando também a distribuição potencial da fauna de vertebrados. Além disso, investigar como atributos funcionais de plantas respondem a esses fatores, com implicações para a diversidade funcional de comunidades vegetais e animais.
\end{itemize}
\end{rSection}

%------------------------------------------------------------------------
%   Financiamento
%------------------------------------------------------------------------

\begin{rSection}{Financiamento} 
\begin{itemize}
\item {\bf Bolsa de Iniciação Científica}, Fundação de Amparo à Pesquisa do Estado de São Paulo (FAPESP) \hfill{\em 2013-2015}\\
{\it Processo}: \href{https://bv.fapesp.br/en/bolsas/142421/effect-of-fragmentation-on-the-persistence-of-anuran-amphibians-amphibia-anura-within-the-atlanti/}{\underline{2013/02883-7}}\\
{\it Valor}: R\$ 10.539,87

\item {\bf Bolsa de Mestrado}, Fundação de Amparo à Pesquisa do Estado de São Paulo (FAPESP) \hfill{\em 2017-2018}\\
{\it Processo}: \href{https://bv.fapesp.br/en/bolsas/172826/effect-of-landscape-modifications-and-climate-changes-on-the-persistence-of-amphibians-in-the-atlant/}{\underline{2017/09676-8}}\\
{\it Valor}: R\$ 16.248,54

\item {\bf Bolsa de Doutorado}, Fundação de Amparo à Pesquisa do Estado de São Paulo (FAPESP) \hfill{\em 2022-2024}\\
{\it Processo}: \href{https://bv.fapesp.br/en/bolsas/203713/landscape-structure-as-a-predictor-of-taxonomic-and-functional-diversity-of-amphibians-in-the-atlant/}{\underline{2022/01899-6}}\\
{\it Valor}: R\$ 91.002,24

\item {\bf Bolsa de Pós-doutorado}, Fundação de Amparo à Pesquisa do Estado de São Paulo (FAPESP) \hfill{\em 2024-presente}\\
{\it Processo}: \href{https://bv.fapesp.br/en/bolsas/223203/impacts-of-edge-effects-fire-and-climate-change-on-vegetation-composition-and-functional-diversity-a/}{\underline{2024/19865-6}}\\
{\it Valor}: R\$ 316.800,00
\end{itemize}
\end{rSection}

%------------------------------------------------------------------------
%   Prêmios e Honrarias
%------------------------------------------------------------------------

\begin{rSection}{Prêmios e honrarias} 
\begin{itemize}
\item {Prêmio de Alto Desempenho Acadêmico, Universidade Estadual Paulista (UNESP)} \hfill{\em 2015}
\item {Prêmio The Harry R. Painton, American Ornithological Society} \hfill{\em 2023}
\end{itemize}
\end{rSection}

%-------------------------------------------------------------------------
%	Publications 
%-------------------------------------------------------------------------

\begin{rSection}{Publicações}

{\bf Artigos revisados por pares (mais relevantes)}

\begin{enumerate} 

\item {\bf Vancine MH}, Muylaert RL, Niebuhr BB, Oshima JEF, Tonetti V, Bernardo R, De Angelo C, Rosa MR, Grohmann CH, Ribeiro MC. 2024. {\bf The Atlantic Forest of South America: spatiotemporal dynamics of remaining vegetation and implications for conservation}. {\it Biological Conservation} 291:110499. \href{https://doi.org/10.1016/j.biocon.2024.110499}{\underline{10.1016/j.biocon.2024.110499}}

\item Gonçalves-Souza T, Chase J, Haddad N, {\bf Vancine MH}, Melo FPL, Aizen M, Bernard E, Chiarello GA, Didham R, Faria D, Gibb H, Lima M, Magnago L, Mariano Neto E, Nogueira A, Nemésio A, Passamani M, Pinho BX, Rocha-Santos L, Rodrigues R, Safar N, Santos B, Soto-Werschitz A, Tabarelli M, Uehara-Prado M, Vasconcelos H, Vieira S, Sanders, N. 2025. {\bf Species turnover does not rescue biodiversity in fragmented landscapes}. {\it Nature} 640:702-706. \href{https://doi.org/10.1038/s41586-025-08688-7}{\underline{10.1038/s41586-025-08688-7}}

\item Beca G, {\bf Vancine MH}, Carvalho CS, Pedrosa F, Alves RSC, Buscariol D, Peres CA, Ribeiro MC, Galetti M. 2017. {\bf High mammal species turnover in forest patches immersed in biofuel plantations}. {\it Biological Conservation} 210:352-359. \href{https://doi.org/10.1016/j.biocon.2017.02.033}{\underline{10.1016/j.biocon.2017.02.033}}

\end{enumerate} 

{\bf Artigos revisados por pares (lista completa - \today)}

\begin{enumerate} 
\item Beca G, {\bf Vancine MH}, Carvalho CS, Pedrosa F, Alves RSC, Buscariol D, Peres CA, Ribeiro MC, Galetti M. 2017. {\bf High mammal species turnover in forest patches immersed in biofuel plantations}. {\it Biological Conservation} 210:352-359. \href{https://doi.org/10.1016/j.biocon.2017.02.033}{\underline{10.1016/j.biocon.2017.02.033}}

\item de Castro Pena JC, Goulart F, Fernandes GW, Hoffmann D, Leite FS, dos Santos NB, Soares-Filho B, Sobral-Souza T, {\bf Vancine MH}, Rodrigues M. 2017. {\bf Impacts of mining activities on the potential geographic distribution of eastern Brazil mountaintop endemic species}. {\it Perspectives in Ecology and Conservation} 15(3):172-178. \href{https://doi.org/10.1016/j.pecon.2017.07.005}{\underline{10.1016/j.pecon.2017.07.005}}

\item Regolin AL, Cherem JJ, Graipel ME, Bogoni JA, Ribeiro JW, {\bf Vancine MH}, Castilho PVD. 2017. {\bf Forest cover influences occurrence of mammalian carnivores within Brazilian Atlantic Forest}. {\it Journal of Mammalogy} 98(6):1721-1731. \href{https://doi.org/10.1093/jmammal/gyx103}{\underline{10.1093/jmammal/gyx103}}

\item Sobral-Souza T, {\bf Vancine MH}, Ribeiro MC, Lima-Ribeiro MS. 2018. {\bf Efficiency of protected areas in Amazon and Atlantic Forest conservation: A spatio-temporal view}. {\it Acta Oecologica} 87:1-7. \href{https://doi.org/10.1016/j.actao.2018.01.001}{\underline{10.1016/j.actao.2018.01.001}}

\item Muylaert RL, {\bf Vancine MH}, Bernardo R, Oshima JEF, Sobral-Souza T, Tonetti VR, Ribeiro MC. 2018. {\bf Uma nota sobre os limites territoriais da Mata Atlântica}. {\it Oecologia Australis} 22(3):302-311. \href{https://doi.org/10.4257/oeco.2018.2203.09}{\underline{10.4257/oeco.2018.2203.09}}

\item {\bf Vancine MH}, Duarte KS, Souza YS, Giovanelli JGR, Sobrinho PMM, López A, Bovo RP, Maffei F, Lion MB, Ribeiro-Júnior JW, Brassaloti R, Ortiz C, Sawakuchi HO, Forti LR, Cacciali P, Bertoluci J, Haddad CFB, Ribeiro MC. 2018. {\bf ATLANTIC AMPHIBIANS: a data set of amphibian communities from the Atlantic Forests of South America}. {\it Ecology} 99(7):1692-1692. \href{https://doi.org/10.1002/ecy.2392}{\underline{10.1002/ecy.2392}}

\item Ferro e Silva AM, Sobral-Souza T, {\bf Vancine MH}, Muylaert RL, Abreu AP, Pelloso SM, Carvalho MDB, Andrade L, Ribeiro MC, Toledo MJ. 2018. {\bf Spatial prediction of risk areas for vector transmission of \textbf{\textit{Trypanosoma cruzi}} in the State of Paraná, southern Brazil}. {\it PLoS Neglected Tropical Diseases} 12(10):e0006907. \href{https://doi.org/10.1371/journal.pntd.0006907}{\underline{10.1371/journal.pntd.0006907}}

\item Bertassoni A, Costa RT, Gouvea JA, Bianchi RDC, Ribeiro JW, {\bf Vancine MH}, Ribeiro MC. 2019. {\bf Land-use changes and the expansion of biofuel crops threaten the giant anteater in southeastern Brazil}. {\it Journal of Mammalogy} 100(2):435-444. \href{https://doi.org/10.1093/jmammal/gyz042}{\underline{10.1093/jmammal/gyz042}}

\item Moraes AM, {\bf Vancine MH}, Moraes AM, Oliveira Cordeiro CL, Pinto MP, Lima AA, Sobral-Souza T. 2019. {\bf Predicting the potential hybridization zones between native and invasive marmosets within Neotropical biodiversity hotspots}. {\it Global Ecology and Conservation} 20:e00706.06. \href{https://doi.org/10.1016/j.gecco.2019.e00706}{\underline{10.1016/j.gecco.2019.e00706}}

\item Santos JP, Sobral‐Souza T, Brown Jr KS, {\bf Vancine MH}, Ribeiro MC, Freitas AV. 2020. {\bf Effects of landscape modification on species richness patterns of fruit‐feeding butterflies in Brazilian Atlantic Forest}. {\it Diversity and Distributions} 26(2):196-208. \href{https://doi.org/10.1111/ddi.13007}{\underline{10.1111/ddi.13007}}

\item Marjakangas E, Abrego N, Grøtan V, Lima RAF, Bello C, Bovendorp RS, Culot L, Hasui E, Muylaert RL, Lima F, Niebuhr B, Oliveira AA, Pereira L, Prado I, Stevens RD, {\bf Vancine MH}, Ribeiro MC, Galetti M, Ovaskainen O. 2020. {\bf Fragmented tropical forests lose mutualistic plant-animal interactions}. {\it Diversity and Distributions} 26(2):154-168. \href{https://doi.org/10.1111/ddi.13010}{\underline{10.1111/ddi.13010}}

\item Bello C, Cintra ALP, Barreto E, {\bf Vancine MH}, Sobral-Souza T, Graham CH, Galetti M. 2021. {\bf Environmental niche and functional role similarity between invasive and native palms in the Atlantic Forest}. {\it Biological Invasions} 23(3):741-754. \href{https://doi.org/10.1007/s10530-020-02400-8}{\underline{10.1007/s10530-020-02400-8}}

\item Gusmão RA, Hernandes FA, {\bf Vancine MH}, Naka LN, Doña J, Gonçalves‐Souza T. 2021. {\bf Host diversity outperforms climate as a global driver of symbiont diversity in the bird‐feather mite system}. {\it Diversity and Distributions} 27(3):416-426. \href{https://doi.org/10.1111/ddi.13201}{\underline{10.1111/ddi.13201}}

\item Da Silveira NS, {\bf Vancine MH}, Jahn AE, Pizo MA, Sobral-Souza T. 2021. {\bf Future climate change will impact the size and location of breeding and wintering areas of migratory thrushes in South America}. {\it The Condor} 123(2):duab006. \href{https://doi.org/10.1093/ornithapp/duab006}{\underline{10.1093/ornithapp/duab006}}

\item Jover A, Cabrera A, Ramos A, {\bf Vancine MH}, Suárez AM, Machell J, Pérez-Lloréns JL. 2021. {\bf Distribution of macroalgae epiphytes and host species from the Cuban marine shelf inferred from ecological modelling}. {\it Aquatic Botany} 172:103395. \href{https://doi.org/10.1016/j.aquabot.2021.103395}{\underline{10.1016/j.aquabot.2021.103395}}

\item Bercê W, Bello C, Mendes CP, {\bf Vancine MH}, Galetti M, Ballari SA. 2021. {\bf Invasive wild boar’s distribution overlap with threatened native ungulate in Patagonia}. {\it Journal of Mammalogy} 102(5):1298-1308. \href{https://doi.org/10.1093/jmammal/gyab099}{\underline{10.1093/jmammal/gyab099}}

\item Oshima JEF, Jorge MLS, Sobral-Souza T, Börger L, Keuroghlian A, Peres CA, {\bf Vancine MH}, Colleni B, Ribeiro MC. 2021. {\bf Setting priority conservation management regions to reverse rapid range decline of a key neotropical forest ungulate}. {\it Global Ecology and Conservation} 31:e01796. \href{https://doi.org/10.1016/j.gecco.2021.e01796}{\underline{10.1016/j.gecco.2021.e01796}}

\item Monteiro ECS, Pizo MA, {\bf Vancine MH}, Ribeiro MC. 2022. {\bf Forest cover and connectivity have pervasive effects on the maintenance of evolutionary distinct interactions in seed dispersal networks}. {\it Oikos} 2022(2):oik.08240. \href{https://doi.org/10.1111/oik.08240}{\underline{10.1111/oik.08240}}

\item Ribeiro-Souza P, Graipel ME, Astúa D, {\bf Vancine MH}, Pires JSR. 2022. {\bf Effects of climate change on distribution and areas that protect two neotropical marsupials associated with aquatic environments}. {\it Ecological Informatics} 68:101570. \href{https://doi.org/10.1016/j.ecoinf.2022.101570}{\underline{10.1016/j.ecoinf.2022.101570}}

\item Costa MF, Francisconi AF, {\bf Vancine MH}, Zucchi MI. 2022. {\bf Climate change impacts on the \textbf{\textit{ Copernicia alba}} and \textbf{\textit{Copernicia prunifera}} (Arecaceae) distribution in South America}. {\it Brazilian Journal of Botany} 45:807-818. \href{https://doi.org/10.1007/s40415-022-00801-8}{\underline{10.1007/s40415-022-00801-8}}

\item Muylaert RL, Kingston T, Luo J, {\bf Vancine MH}, Galli N, Carlson CJ, John RS, Rulli MC, Hayman DT. 2022. {\bf Present and future distribution of bat hosts of sarbecoviruses: implications for conservation and public health}. {\it Proceedings of the Royal Society B} 289(1975):20220397. \href{https://doi.org/10.1098/rspb.2022.0397}{\underline{10.1098/rspb.2022.0397}}

\item Borges GA, Mancilla G, Siqueira AB, {\bf Vancine MH}, Ribeiro MC, Maia JCS. 2022. {\bf The fate of vegetation remnants in the southern Amazon’s largest threatened hotspot: part (I) a 33-year analysis of LULCC in the Tapajos River basin, Brazil}. {\it Research, Society and Development} 11(10):e448111032553. \href{https://doi.org/10.33448/rsd-v11i10.32553}{\underline{10.33448/rsd-v11i10.32553}}

\item Galetti M, Carmignotto AP, Percequillo AR, Santos MCO, Ferraz KMPMB, Lima F, {\bf Vancine MH}, Muylaert RL, Bonfim FCG, Magioli M, Abra FD, Chiarello AG, Duarte JMB, Morato R, de Mello Beisiegel B, Olmos F, Galetti Jr. PM, Ribeiro MC. 2022. {\bf Mammals in São Paulo State: diversity, distribution, ecology, and conservation}. {\it Biota Neotropica} 22(spe):e20221363. \href{https://doi.org/10.1590/1676-0611-bn-2022-1363}{\underline{10.1590/1676-0611-bn-2022-1363}}

\item Santos PM, Ferraz KMPMB, Ribeiro MC, Niebuhr BB, {\bf Vancine MH}, Chiarello AC, Paglia AP. 2022. {\bf Natural forest regeneration on anthropized landscapes could overcome climate change effects on the endangered maned sloth (\textbf{\textit{ Bradypus torquatus}}, Illiger 1811)}. {\it Journal of Mammology} 103(6):1383-1396. \href{https://doi.org/10.1093/jmammal/gyac084}{\underline{10.1093/jmammal/gyac084}}

\item Dutra VAB, {\bf Vancine MH}, Lima AAM, Toledo PM. 2023. {\bf Dinâmica da paisagem e fragmentação de ecossistemas em três bacias hidrográficas na Amazônia Oriental entre 1985 e 2019}. {\it Revista Brasileira de Geografia Física} 16(02):936-949. \href{https://doi.org/10.26848/rbgf.v16.2.p936-949}{\underline{10.26848/rbgf.v16.2.p936-949}}

\item Amaral IS, Pereira JB, {\bf Vancine MH}, Morales AE, Althoff SL, Gregorin R, Pereira MJR, Valiali VH, Oliveira LR. 2023. {\bf Where do they live? Predictive geographic distribution of \textbf{\textit{Tadarida brasiliensis brasiliensis}} (Chiroptera, Molossidae) in South America}. {\it Neotropical Biology and Conservation} 18(3):139-156. \href{https://doi.org/10.3897/neotropical.18.e101390}{\underline{10.3897/neotropical.18.e101390}}

\item Anunciação PA, Ernst R, Martello F, {\bf Vancine MH}, Ribeiro MC, Carvalho LMT. 2023. {\bf Climate-driven loss of taxonomic and functional richness in Atlantic Forest anurans}. {\it Perspectives in Ecology and Conservation} 21(4):274-285. \href{https://doi.org/10.1016/j.pecon.2023.09.001}{\underline{10.1016/j.pecon.2023.09.001}}

\item Teixeira JVS, Bonfim FCG, {\bf Vancine MH}, Ribeiro MC, Oliveira LC. 2023. {\bf Effect of landscape attributes at multiple scales on the occurrence of the threatened golden-headed lion tamarin, \textbf{\textit{Leontopithecus chrysomelas}} Kuhl, 1820 (Primates, Callitrichidae)}. {\it American Journal of Primatology} 86(4):e23588. \href{https://doi.org/10.1002/ajp.23588}{\underline{10.1002/ajp.23588}}

\item Tonetti V, Bocalini F, Schunck F, {\bf Vancine MH}, Butti M, Ribeiro MC, Pizo M, Balmford A. 2024. {\bf The Protected Areas network may be inefficient to cover biodiversity in a fragmented tropical hotspot under different climate scenarios}. {\it Perspectives in Ecology and Conservation} 22(1):63-71. \href{https://doi.org/10.1016/j.pecon.2023.12.002}{\underline{10.1016/j.pecon.2023.12.002}}

\item {\bf Vancine MH}, Muylaert RL, Niebuhr BB, Oshima JEF, Tonetti V, Bernardo R, De Angelo C, Rosa MR, Grohmann CH, Ribeiro MC. 2024. {\bf The Atlantic Forest of South America: spatiotemporal dynamics of remaining vegetation and implications for conservation}. {\it Biological Conservation} 291:110499. \href{https://doi.org/10.1016/j.biocon.2024.110499}{\underline{10.1016/j.biocon.2024.110499}}

\item Barbosa FS, {\bf Vancine MH}, Ribeiro MC, Siminski A. 2024. {\bf Análise espacial da fragmentação da paisagem: um estudo de caso sobre os fragmentos florestais destinados à Reserva Legal em duas bacias hidrográficas no Estado do Piauí}. {\it Revista Observatório de la Economía Latinoamericana} 22(11):e7996. \href{https://doi.org/10.55905/oelv22n11-247}{\underline{10.55905/oelv22n11-247}}

\item Marques RCM, {\bf Vancine MH}, Súarez YR, Pereira JG, Domingos JD, Silva ABB, Pereira, ZV. 2025. {\bf Dinâmica espaço temporal: variações na composição e configuração da vegetação}. {\it Revista Brasileira de Geografia Física} 18(2):1334-1348. \href{https://doi.org/10.26848/rbgf.v18.2.p1334-1348}{\underline{10.26848/rbgf.v18.2.p1334-1348}}

\item Carvalho T, Falconi N, White T, Anjos LA, Giasson LOM, {\bf Vancine MH}, Haddad CFB, Toledo LF, Becker CG. 2025. {\bf The role of seasonal migration in predicting amphibian population persistence across fragmented tropical landscapes: an Individual-Based Model}. {\it Biodiversity and Conservation} 34:1291-1310. \href{https://doi.org/10.1007/s10531-025-03016-x}{\underline{10.1007/s10531-025-03016-x}}

\item Gonçalves-Souza T*, {\bf Vancine MH*}, Sanders NJ, Haddad NM, (…), Chase JM. 2025. LandFrag: a dataset to investigate the effects of habitat loss and fragmentation on biodiversity in forest fragments. {\it Global Ecology and Biogeography (Data Article)} 34(2):e70015. \href{https://doi.org/10.1111/geb.70015}{\underline{10.1111/geb.70015}}

\item Gonçalves-Souza T, Chase J, Haddad N, {\bf Vancine MH}, Melo FPL, Aizen M, Bernard E, Chiarello GA, Didham R, Faria D, Gibb H, Lima M, Magnago L, Mariano Neto E, Nogueira A, Nemésio A, Passamani M, Pinho BX, Rocha-Santos L, Rodrigues R, Safar N, Santos B, Soto-Werschitz A, Tabarelli M, Uehara-Prado M, Vasconcelos H, Vieira S, Sanders, N. 2025. {\bf Species turnover does not rescue biodiversity in fragmented landscapes}. {\it Nature} 640:702-706. \href{https://doi.org/10.1038/s41586-025-08688-7}{\underline{10.1038/s41586-025-08688-7}}

\item Alves-Ferreira G*, {\bf Vancine MH*}, Mota FMM, Bello C, Sobral-Souza T, Percequillo AR, Lacher Jr TE, Galetti M, Bovendorp RS. 2025. {\bf From hot to cold spots: climate change will modify endemism centers of small mammals in a biodiversity hotspot}. {\it Diversity and Distributions} 31(5):e70026. \href{https://doi.org/10.1111/ddi.70026}{\underline{10.1111/ddi.70026}} 

\item Goebel LGA, {\bf Vancine MH}, Bogoni JA, Longo GR, Calicis M, Fearside PM, Palmeirim AF, Santos-Filho M. 2025. {\bf The impact of Amazon deforestation is magnified by changing the configuration of forest cover}. {\it Environmental Conservation}. \href{https://doi.org/10.1017/S0376892925000086}{\underline{10.1017/S0376892925000086}}

\end{enumerate} 

*Primeira coautoria

{\bf Livos}

\begin{enumerate} 
\item Da Silva FR, Gonçalves-Souza T, Paterno GB, Provete DB, {\bf Vancine MH}. 2022. {\bf Análises Ecológicas no R}. Nupeea: Recife, PE, Canal 6: São Paulo. 640 p. ISBN 978-85-7917-564-0. \href{https://analises-ecologicas.com/}{\underline{link}}
\end{enumerate} 

{\bf Para todas as publicações, citações e revisores de periódicos}
\\\href{https://orcid.org/0000-0001-9650-7575}{\underline{ORCID}}, \href{https://www.webofscience.com/wos/author/record/837504}{\underline{Web of Science}}, \href{https://www.scopus.com/authid/detail.uri?authorId=57193451888}{\underline{Scopus}}, \href{https://scholar.google.com/citations?user=i-2xZBQAAAAJ}{\underline{Google Scholar}}

\end{rSection}

%-------------------------------------------------------------------------
%	Manuscripts under review
%-------------------------------------------------------------------------

\begin{rSection}{Manuscritos em revisão}

\begin{enumerate} 

\item {\bf Vancine MH*}, Niebuhr BB*, Muylaert RL, Oshima JEF, Tonetti V, Bernardo R, Alves RSC, Zanette EM, Souza VC, Giovanelli JGR, Grohmann CH, Galetti M, Ribeiro MC. {\bf ATLANTIC SPATIAL: a data set of landscape, topographic, hydrological and anthropogenic metrics for the Atlantic Forest}. {\it Ecology (Data Paper)}. Preprint: {\it EcoEvoRxiv}. \href{https://doi.org/10.32942/X26P58}{\underline{10.32942/X26P58}}

\item {\bf Vancine MH}, Dodonov P, Vilela B, Diele-Viegas L, Souza VC, Nunes FAS, Silva AJO, Milz B, Kita CA, Mello MAR, Muylaert RL. The challenges and nuances of teaching the R programming language to ecologists. {\it Oecologia Australis}.

\item da Silva RFB, Millington JDA, Yue D, {\bf Vancine MH}, Magnago LFS, Viña A, Bin F, Huesca M, Viera SA, Garibaldi LA, Liu J. Secondary natural vegetation gains in the Atlantic Forest do not offset losses of carbon stocks and conservation of priority areas. {\it Biological Conservation}.

\item Francisconi AF, {\bf Vancine MH}, Zucchi MI. Expansion of açaí fruit and heart-of-palm production promotes hybridization and introgression in {\it Euterpe} palms. {\it BMC Plant Biology}.

\end{enumerate} 

*Primeira coautoria

\end{rSection}


%-------------------------------------------------------------------------
%   Habilidades
%-------------------------------------------------------------------------

\begin{rSection}{Habilidades}

{\bf Linguagens de programação}
\begin{itemize}
\item R (avançado, incluindo tidyverse, markdown, quarto), Shiny (básico), git (intermediário), LaTeX (intermediário), Python (básico), Julia (básico), JavaScript (básico), shell/bash (básico), HTML/CSS (básico)
\end{itemize}

{\bf Softwares, aplicativos e plataformas}
\begin{itemize}
\item QGIS (avançado), GRASS GIS (avançado), GNU/Linux (intermediário), Fragstats (intermediário), GuidosToolbox (intermediário), LibreOffice suite (intermediário), Google Earth Engine (básico), GeoDa (básico), ArcGIS (básico), Inkscape (básico), Sublime Text (básico), Emacs (básico)
\end{itemize}

{\bf Pacotes em R}
\begin{itemize} 
\item {\it lsmetrics}: fornece cálculo multiescalar e espacialização de métricas de paisagem usando GRASS GIS\\
Autores: Bernardo B. Niebuhr*, {\bf Maurício Humberto Vancine*}, Felipe Martello Ribeiro, Renata L. Muylaert, John Wesley Ribeiro, Milton Cezar Ribeiro\\
Link: \href{https://mauriciovancine.github.io/lsmetrics}{\underline{mauriciovancine.github.io/lsmetrics}}

\item {\it atlanticr}: fornece dados do ATLANTIC, data papers da Mata Atlântica\\
Autores: {\bf Maurício Humberto Vancine}, Bernardo Niebuhr, Renata L. Muylaert, Mauro Galetti, Milton Cezar Ribeiro\\
Link: \href{https://mauriciovancine.github.io/atlanticr}{\underline{mauriciovancine.github.io/atlanticr}}

\item {\it amphiBR}: conjunto de dados da publicação oficial da Lista de Anfíbios do Brasil publicada pela Sociedade Brasileira de Herpetologia\\
Autores: Paulo Barros de Abreu Junior, {\bf Maurício Humberto Vancine}, Diogo B. Provete\\
Link: \href{https://paulobarros.github.io/amphiBR}{\underline{paulobarros.github.io/amphiBR}}

\item {\it ecodados}: base de dados ecológicos para ensino de estatística\\
Autores: Gustavo Paterno, Diogo B. Provete, Fernando Rodrigues da Silva, Thiago Gonçalves-Souza, {\bf Maurício Humberto Vancine}\\
Link: \href{https://paternogbc.github.io/ecodados/}{\underline{paternogbc.github.io/ecodados}}

\item {\it lsma}: análise multiescalar da estrutura da paisagem usando R\\
Autores: Wilson Frantine-Silva, Lazaro S. Carneiro, André Luis Regolin, {\bf Maurício Humberto Vancine}, Juliana S. Santos, Bernardo Brandão Niebuhr, Edson Valmorbida, Maria Cristina Gaglianone, Milton C. Ribeiro\\
Link: \href{https://github.com/wilsonfrantine/landscapeDecoupler}{\underline{wilsonfrantine/lsma}}
\end{itemize} 

*Primeira coautoria

{\bf Conhecimentos estatísticos}
\begin{itemize} 
\item {\it Estatística descritiva}: manipulação de dados tabulares (avançado), manipulação de dados espaciais (avançado), visualização de dados tabulares (avançado), visualização de dados espaciais (avançado).

\item {\it Estatística univariada}: métodos frequentistas (avançado), métodos baseados em verossimilhança (intermediário), seleção de modelos (intermediário), inferência Bayesiana (básico).

\item {\it Estatística multivariada}: métodos de ordenação (intermediário), métodos de agrupamento (intermediário), análise discriminante (básico).
\end{itemize}

{\bf Idiomas}
\begin{itemize}
\item Inglês (intermediário)
\item Espanhol (básico)
\item Italiano (básico)
\item Português (nativo)
\end{itemize}

\end{rSection}


%-------------------------------------------------------------------------
%   References
%-------------------------------------------------------------------------

\begin{rSection}{Referências}
\begin{itemize}
\item {\bf Dr. Milton C. Ribeiro, Universidade Estadual Paulista (UNESP), Brasil}\\
milton.c.ribeiro@unesp.br, +55-19-3526-9680
\item {\bf Dr. Célio F. B. Haddad, Universidade Estadual Paulista (UNESP), Brasil}\\
haddad1000@gmail.com, +55-19-3526-4302
\item {\bf Dr. Mathias Pires Mistretta, Universidade Estadual de Campinas (Unicamp), Brasil}\\
mathiasmpires@gmail.com, +55-19-3521-6275
\item {\bf Dr. Thiago Gonçalves-Souza, Universidade de Michigan, EUA}\\
tgoncalv@umich.edu, +1-734-596-0508
\item {\bf Dr. Renata L. Muylaert, Disease Ecology Lab, The University of Sydney, Austrália}\\
R.deLaraMuylaert@massey.ac.nz, +64-06-356-9099 ext. 85217
\item {\bf Dr. Bernardo B. Niebuhr, Norwegian Institute for Nature Research (NINA), Noruega}\\
bernardo.brandao@nina.no, +47-406-41-783
\item {\bf Dr. João G. R. Giovanelli, Seleção Natural, Brasil}\\
joao@selecaonatural.net, +47-19-991-037-348
\end{itemize}
\end{rSection}

\end{document}----------------------------